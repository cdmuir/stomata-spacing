% Options for packages loaded elsewhere
\PassOptionsToPackage{unicode}{hyperref}

% Set document class options
\documentclass[webpdf,large,modern,unnumsec,namedate]{oup-authoring-template}

% one column
\onecolumn

%\usepackage{showframe}

% line numbers

% use upquote if available, for straight quotes in verbatim environments
\IfFileExists{upquote.sty}{\usepackage{upquote}}{}

% From Pandoc template for its feature
\usepackage{xcolor}
\usepackage{hyperref}

\hypersetup{
  pdftitle={Does stomatal patterning in amphistomatous leaves minimize the CO\_2 diffusion path length with leaves of Arabidopsis thaliana?},
  pdfkeywords={amphistomy, Arabidopsis thaliana, CO\(_2\)
diffusion, finite element method, optimality, photosynthesis, stomata},
  breaklinks=true,
  bookmarks=true,
  hidelinks,
  pdfcreator={LaTeX via pandoc}}



% tightlist command for lists without linebreak
\providecommand{\tightlist}{%
  \setlength{\itemsep}{0pt}\setlength{\parskip}{0pt}}




% Counters for addresses and footnotes
\newcounter{correspcnt} % For author footnotes
\renewcommand*{\thecorrespcnt}{\fnsymbol{correspcnt}}
\newcounter{addrcnt} % For author addresses

% Macros for dealing with affiliations, footnotes, etc.
\makeatletter

\def\MyNewLabel#1#2#3{\expandafter\gdef\csname #1@#2\endcsname{#3}}

\def\MyRef#1#2{\@ifundefined{#1@#2}{???}{\csname #1@#2\endcsname}}

\newcommand*\ifcounter[1]{%
  \ifcsname c@#1\endcsname
    \expandafter\@firstoftwo
  \else
    \expandafter\@secondoftwo
  \fi
}

\newcommand*\addrlblbycode[1]{%
  \ifcounter{ADDRLBL@#1}
    {}
    {\refstepcounter{addrcnt}\newcounter{ADDRLBL@#1}\setcounter{ADDRLBL@#1}{\value{addrcnt}}}%
    \arabic{ADDRLBL@#1}%
}

\newcommand*\addrbycode[1]{%
  \ifcounter{ADDR@#1}
    {}
    {\newcounter{ADDR@#1}%
     \address[\addrlblbycode{#1}]{\MyRef{ADDRTXT}{#1}}}%
}

\newcommand*\corresplblbycode[1]{%
  \ifcounter{CORRESPLBL@#1}
    {}
    {\refstepcounter{correspcnt}\newcounter{CORRESPLBL@#1}\setcounter{CORRESPLBL@#1}{\value{correspcnt}}}%
    \fnsymbol{CORRESPLBL@#1}%
}

\newcommand*\correspbycode[1]{%
  \ifcounter{CORRESP@#1}
    {}
    {\newcounter{CORRESP@#1}%
     \corresp[\corresplblbycode{#1}]{\MyRef{CORRESPTXT}{#1}}}%
}

\makeatother

% Add missing \city command mentioned in documentation but absent from cls
\providecommand\city[1]{#1}

% Create labels for Addresses if the are given in Elsevier format
   \MyNewLabel{ADDRTXT}{UHM}{%
  School of Life Sciences, University of Hawaiʻi at Mānoa, Honolulu, HI
96822%
 }
   \MyNewLabel{ADDRTXT}{CU}{%
  Ecology and Evolutionary Biology, Universtity of Colorado, Boulder, CO
80309%
 }
   \MyNewLabel{ADDRTXT}{NIAB}{%
  Department of Crop Science and Production Systems, NIAB, Cambridge,
CB3 0LE, UK%
 }
   \MyNewLabel{ADDRTXT}{UCD}{%
  Department of Plant Sciences, University of California, Davis, CA
95616%
 }
   \MyNewLabel{ADDRTXT}{UWM}{%
  Department of Botany, University of Wisconsin, Madison, WI 53706%
 }

% Create labels for Footnotes if they are given in Elsevier format

% Pandoc header-include feature
\usepackage[nomarkers,tablesfirst]{endfloat}
\usepackage{lineno}
\linenumbers
\usepackage{hyperref}
\renewcommand{\figureautorefname}{Fig.}
\usepackage[detect-none]{siunitx}
\sisetup{range-phrase = \text{--}}
\usepackage{caption}
\usepackage{newunicodechar,graphicx}
\DeclareRobustCommand{\okina}{\raisebox{\dimexpr\fontcharht\font`A-\height}{\scalebox{0.8}{`}}}
\newunicodechar{ʻ}{\okina}
% Pandoc header-include feature
\usepackage{booktabs}

\begin{document}

\journaltitle{Journal Title Here}
\DOI{DOI HERE}
\copyrightyear{YYYY}
\pubyear{YYYY}
\access{Advance Access Publication Date: Day Month Year}
\appnotes{Paper}

\firstpage{1}



\title[]{Does stomatal patterning in amphistomatous leaves minimize the
CO\(_2\) diffusion path length with leaves of \emph{Arabidopsis
thaliana}?}

\newcounter{thisauthcorresp} % For storage if author is corresponding author
\newcounter{thisauththanks} % For storage if author has thanks



\author[%
\addrlblbycode{UHM},\addrlblbycode{CU}%
,\refstepcounter{correspcnt}\setcounter{thisauthcorresp}{\value{correspcnt}}\fnsymbol{thisauthcorresp}%
%
%
]{Jacob L. Watts}

\addrbycode{UHM}
\addrbycode{CU}

\corresp[\fnsymbol{thisauthcorresp}]{Corresponding author. \href{mailto:Jacob.Watts-1@colorado.edu}{\nolinkurl{Jacob.Watts-1@colorado.edu}}}




\author[%
\addrlblbycode{NIAB}%
%
%
%
]{Graham J. Dow}

\addrbycode{NIAB}






\author[%
\addrlblbycode{UCD}%
%
%
%
]{Thomas N. Buckley}

\addrbycode{UCD}






\author[%
\addrlblbycode{UHM},\addrlblbycode{UWM}%
%
%
%
]{Christopher D. Muir}

\addrbycode{UHM}
\addrbycode{UWM}






% Add author mark
\authormark{Jacob L. Watts et al.}

\received{Date}{0}{Year}
\revised{Date}{0}{Year}
\accepted{Date}{0}{Year}

%\editor{Associate Editor: Name}

\abstract{
We will write abstract after Discussion and Conclusions are complete}

\keywords{amphistomy; Arabidopsis thaliana; CO\(_2\) diffusion; finite
element method; optimality; photosynthesis; stomata}


\maketitle


\hypertarget{introduction}{%
\section{Introduction}\label{introduction}}

Stomatal anatomy (e.g.~size, density, distribution, and patterning) and
movement regulate gas exchange during photosynthesis, namely CO\(_2\)
assimilation and water loss through transpiration. Since waxy cuticles
are mostly impermeable to CO\(_2\) and H\(_2\)O, stomata are the primary
entry points through which gas exchange occurs despite making up a small
percentage of the leaf area \citep{lange_responses_1971}. Stomata
consist of two guard cells which open and close upon changes in turgor
pressure or hormonal cues \citep{mcadam_linking_2016}. The stomatal pore
leads to an internal space known as the substomatal cavity where gases
contact the mesophyll. Once in the mesophyll, CO\(_2\) diffuses
throughout a network of intercellular air space (IAS) and into mesophyll
cells where CO\(_2\) assimilation (\(A\)) occurs within the chloroplasts
\citep{lee_diffusion_1964}. Stomatal conductance and transpiration are
determined by numerous environmental and anatomical parameters such as
vapor pressure deficit (VPD), irradiance, temperature, wind speed, leaf
water potential, IAS geometry, mesophyll cell anatomy, and stomatal
anatomy.

Many successful predictions about stomata and other leaf traits can be
made by hypothesizing that natural selection should optimizes CO\(_2\)
gain per unit of water loss
\citep{cowan_stomatal_1977, buckley_optimal_2017, sperry_predicting_2017}.
However, stomatal anatomy may be partially constrained by physical and
developments limits on phenotypic expression
\citep{croxdale_stomatal_2000, harrison_influence_2020, muir_how_2023}.
Sometimes optimization leads to similar phenotypes across many disparate
species. For example, almost all stomata follow the one cell spacing
rule to maintain proper stomatal functioning
\citep{geisler_oriented_2000, dow_physiological_2014}; however some
species (notably in \emph{Begonia}) appear to benefit from overlapping
vapor shells caused by stomatal clustering
\citep{yi_gan_stomatal_2010, lehmann_effects_2015, papanatsiou_stomatal_2017}.
Stomatal traits also vary adaptively in different environments. Stomatal
density positively co-varies with irradiance during leaf development and
negatively co-varies with CO\(_2\) concentration
\citep{gay_influence_1975, schoch_dependence_1980, woodward_stomatal_1987, royer_stomatal_2001},
consistent with optimality predictions. Stomatal size is jointly
controlled by genome size, light, and stomatal density
\citep{jordan_environmental_2015}. Size positively co-varies with genome
size \citep{roddy_scaling_2020} and negatively co-varies with stomatal
density \citep{camargo_density_2011}. Total stomatal area (size
\(\times\) density) is optimized for operational conductance
(\(g_\text{s,op}\)) rather than maximum conductance (\(g_\text{s,max}\))
such that stomatal apertures are most responsive to changes in the
environment at their operational aperture
\citep{franks_physiological_2012, liu_scaling_2021}. Stomatal aperture
can compensate for maladaptive stomatal densities to an extent
\citep{bussis_stomatal_2006}, but stomatal density and size ultimately
determine a leaf's theoretical \(g_\text{s,max}\)
\citep{sack_developmental_2016}, which is proportional to
\(g_\text{s,op}\) \citep{murray_consistent_2020}. Additionally, low
stomatal densities lead to irregular and insufficient CO\(_2\) supply
and reduced photosynthetic efficiency in areas far from stomata
\citep{pieruschka_lateral_2006, morison_lateral_2005}, while high
stomatal densities can reduce water use efficiency (WUE)
\citep{bussis_stomatal_2006} and incur excessive metabolic costs
\citep{deans_optimization_2020}. In most species, stomata occur on the
abaxial (usually lower) leaf surface; but amphistomy, the occurrence of
stomata on both abaxial and adaxial leaf surfaces, is also prevalent in
high light environments with constant or intermittent access to
sufficient water
\citep{mott_adaptive_1982, jordan_using_2014, muir_light_2018, drake_two_2019, muir_is_2019}.
Amphistomy effectively halves the CO\(_2\) diffusion path length and
boundary layer resistance by doubling boundary layer conductance
\citep{parkhurst_adaptive_1978, harrison_influence_2020, mott_amphistomy_1991}.
Historically, stomatal patterning in dicot angiosperms was thought to be
random with an exclusionary distance surrounding each stomate
\citep{sachs_developmental_1974}; however, the developmental controls of
stomatal patterning are poorly understood and likely more complex than
random development along the leaf surface.
\citet{croxdale_stomatal_2000}{]} reviews three developmental theories
which attempt to explain stomatal patterning in angiosperms: inhibition,
cell lineage, and cell cycle, ultimately arguing for a cell cycle based
control of stomatal patterning.

The patterning and spacing of stomata on the leaf affects photosynthesis
in \(C_3\) leaves by altering the CO\(_2\) diffusion path length from
stomata to sites of carboxylation in the mesophyll. Maximum
photosynthetic rate (\(A_\text{max}\)) in \(C_3\) plants is generally
co-limited by biochemistry and diffusion, but modulated by light
availability
\citep{parkhurst_intercellular_1990, manter_ci_2004, carriqui_diffusional_2015}.
Low light decreasing CO\(_2\) demand by limiting electron transport
rate, leading to relatively high internal CO\(_2\) concentration
(\(C_\text{i}\)) and low \(A_\text{max}\) \citep{kaiser_metabolic_2016}.
In contrast, well hydrated leaves with open stomata in high light,
photosynthesis is often limited by CO\(_2\) supply as resistances from
the boundary layer, stomatal pore, and mesophyll can result in
insufficient CO\(C_2\) supply at the chloroplast to maxmimize
photosynthesis \citep{farquhar_biochemical_1980, lehmeier_cell_2017}. In
this study, we focus primarily on how stomatal patterning affects
diffusion, ignoring boundary layer and mesophyll resistances.

To maximize CO\(_2\) supply from the stomatal pore to chloroplasts,
stomata should be uniformly distributed in an equilateral triangular
grid on the leaf surface so as to minimize stomatal number and CO\(_2\)
diffusion path length \citep{parkhurst_diffusion_1994}. As the diffusion
rate of CO\(_2\) though liquid is approximately \(10^4\times\) slower
than CO\(_2\) diffusion through air, mesophyll resistance is generally
thought to be primarily limited by liquid diffusion
\citep{aalto_three-dimensional_2002, evans_resistances_2009}, but
diffusion through the IAS has also been shown to be a rate limiting
process because the tortuous, disjunct nature of the IAS can greatly
increase diffusion path lengths \citep{harwood_understanding_2021}.
Additionally, tortuosity is higher in horizontal directions (parallel to
leaf surface) than vertical directions (perpendicular to leaf surface)
because of the cylindrical shape and vertical arrangement of pallisade
mesophyll cells \citep{earles_beyond_2018, harwood_understanding_2021}.
However, the ratio of lateral to vertical diffusion rate is still
largely unknown
\citep{morison_lateral_2005, pieruschka_lateral_2005, pieruschka_lateral_2006}.
Depending on the thickness of the leaf, porosity of the leaf mesophyll,
tortuosity of the IAS, and lateral to vertical diffusion rate ratio,
minimizing diffusion path length for CO\(_2\) via optimally distributed
stomata may yield significant increases in CO\(_2\) supply for
photosynthesis and higher \(A_\text{max}\).

We hypothesized that natural selection will favor stomatal patterning
and distribution to minimize the diffusion path length. In
amphistomatous leaves, this would be accomplished by 1) a dispersed,
equilateral triangular distribution of stomata on both abaxial and
adaxial leaf surfaces and 2) coordinated stomatal spacing on each
surface that offsets the position of stomata
(\autoref{fig:ideal-amphi-grid}). Coordination between leaf surfaces is
defined in this study as the occurrence of stomata in areas farthest
from stomata on the opposite leaf surface. Additionally, because
CO\(_2\) is more limiting for photosynthesis under high light, we
hypothesize that in high light 3) there should be more stomata, and 4)
stomata should be more uniformly distributed than in low light. Finally,
as stomatal densities are selected for optimal operational aperture, we
hypothesize that 5) stomatal length will be positively correlated with
the area of the leaf surface to which it is closest. We refer to this as
the `stomatal zone', the leaf area surrounding a focal stomate closest
to that stomate and therefore the zone it supplies with CO\(_2\)). This
way, each stomate can be optimally sized relative to the mesophyll
volume it supplies.

\begin{figure}[ht]
\includegraphics[width = 4in]{figures/ideal-amphi-grid.pdf}
\caption{Idealized amphistomatous stomatal grid with uniform stomatal patterning and perfect abaxial-adaxial coordination.}
\label{fig:ideal-amphi-grid}
\end{figure}

To test these hypotheses, we grew the model plant \emph{Arabidopsis
thaliana} in high, medium, and low light and measured stomatal density,
size, and patterning on both leaf surfaces, and spatial coordination
between them. We use Voronoi tessellation techniques to calculate
stomatal zones. We also used a 2-D porous medium approximation of
CO\(_2\) diffusion and photosynthesis to predict the photosynthetic
advantage of optimal versus suboptimal coordination in stomatal
coordination between surfaces. Specifically, we predicted that traits
which affect diffusion path length (leaf thickness, stomatal density,
leaf porosity, lateral-vertical diffusion rate ratio), diffusion rate
(temperature, pressure), and CO\(_2\) demand (Rubisco concentration,
light) would modulate the advantage of optimal stomatal arrangement
following the relationships outlined in \autoref{tab:hypotheses}. Here,
we integrate over reasonable parameter space to determine the
ecophysiological context most likely to favor stomatal spatial
coordination in amphistomatous leaves.

\begin{table}[ht]
\centering
\begin{tabular}{ll}
  \hline
trait & relationship \\ 
  \hline
leaf thickness & + \\ 
  interstomatal distance & + \\ 
  leaf porosity & - \\ 
  light & + \\ 
   \hline
\end{tabular}
\caption{A summary of the hypothesized relationships between leaf traits and environmental conditions and photosynthetic advantage of stomatal spatial coordination in amphistomatous leaves.} 
\label{tab:hypotheses}
\end{table}

\hypertarget{materials-and-methods}{%
\section{Materials and methods}\label{materials-and-methods}}

\hypertarget{data-preparation}{%
\subsection{Data Preparation}\label{data-preparation}}

Plant material, growth conditions, and three-dimensional confocal
imaging are described in \citet{dow_disruption_2017}. Briefly, Columbia
(Col-0) ecotype of \emph{Arabidopsis thaliana} (L.) Heynh. plants were
grown in three different light environments: low light (PAR = 50
\(\mu \text{mol}~\text{m}^{-2}~\text{s}^{-1}\)), medium light (100
\(\mu \text{mol}~\text{m}^{-2}~\text{s}^{-1}\)), and high light (200
\(\mu \text{mol}~\text{m}^{-2}~\text{s}^{-1}\)). PAR stands for
photosynthetically active radiation. Seeds were surface-sterilized and
stratified at 4°C for 3--5 d in 0.15\% agarose solution and then sown
directly into Pro-Mix HP soil (Premier Horticulture; Quakerstown, PA,
USA) and supplemented with Scott's Osmocote Classic 14-14-14 fertilizer
(Scotts-Sierra, Marysville, OH, USA). At 10--14 d, seedlings were
thinned so only one seedling per container remained. Plants were grown
to maturity in growth chambers where the conditions were as follows: 16
: 8 h, 22 : 20°C, day : night cycle. Imaging of the epidermis and
internal leaf structures was performed using a Leica SP5 confocal
microscope with the protocol developed by
\citet{wuyts_high-contrast_2010} with additional modification described
in \citet{dow_disruption_2017}. We captured 132 images in total, making
66 abaxial-adaxial image pairs. We measured stomatal position and size
using ImageJ \citep{schneider_nih_2012}.

\hypertarget{single-surface-analyses}{%
\subsection{Single surface analyses}\label{single-surface-analyses}}

We tested whether stomata are non-randomly distributed by comparing the
observed stomatal patterning to a random uniform pattern. For each leaf
surface image with \(n\) stomata we generated \(10^3\) synthetic
surfaces with \(n\) stomata uniformly randomly distributed on the
surface. For each sample image, we compared the observed Nearest
Neighbor Index (\(\mathrm{NNI}\)) to the null distribution of
\(\mathrm{NNI}\) values calculated from the synthetic data set.
\(\mathrm{NNI}\) is the ratio of observed mean distance
(\(\overline{D}_O\)) to the expected mean distance (\(\overline{D}_E\))
where \(\overline{D}_E\) is:

\begin{equation}\label{eq:emd}
  \overline{D}_E = \frac{0.5}{\sqrt{A_\text{leaf} / n_\text{stomata}}}.
\end{equation}

\noindent \(A_\text{leaf}\) is leaf area visible in the sampled field
and \(n_\text{stomata}\) the number of stomata. \(\overline{D}_E\) is
the theoretical average distance to the nearest neighbor of each stomate
if stomata were uniformly randomly distributed
\citep{clark_distance_1954}. \(\overline{D}_O\) calculated for each
synthetic data set is:

\begin{equation}\label{eq:omd}
  \overline{D}_O = \frac{\sum_{i=1}^{n_\text{stomata}}d_i}{n_\text{stomata}},
\end{equation}

\noindent where \(d_i\) is the distance between \(\text{stomate}_i\) and
its nearest neighbor. We calculated \(\mathrm{NNI}\) using the \emph{R}
package \textbf{spatialEco} version 2.0.1 \citep{evans_spatialeco_2023}.
The observed stomatal distribution is dispersed relative to a uniform
random distribution if the observed \(NNI\) is greater than 95\% of the
synthetic \(\mathrm{NNI}\) values (one-tailed test). We adjusted
\(P\)-values to account for multiple comparisons using the
Benjamini-Hochberg \citep{benjamini_controlling_1995} false discovery
rate procedure implemented in the \emph{R} package \textbf{multtest}
version 2.56.0 \citep{wong_multiple_2005}.

For each sample image, we also simulated \(10^3\) synthetic data with
\(n\) stomata ideally dispersed in an equilateral triangular grid. For
these grids, we integrated over plausible stomatal densities and then
conditioned on stomatal grids with exactly \(n\) stomata. The simulated
stomatal count was drawn from a Poisson distribution with the mean
parameter \(\lambda\) drawn from a Gamma distribution with shape \(n\)
and scale 1 (\(\lambda \sim \Gamma(n, 1)\)). \(\Gamma(n, 1)\) is the
posterior distribution of \(\lambda\) with a flat prior distribution.
This allows us to integrate over uncertainty in the stomatal density
from the sample image.

We developed a dispersion index \(\mathrm{DI}\) to quantify how close
observed stomatal distributions are to random uniform versus maximally
dispered in an equilateral triangular grid. \(\mathrm{DI}\) varies from
zero to one, where zero is uniformly random and one is ideally
dispersed:

\begin{equation}\label{eq:disp}
  \mathrm{DI} = \frac{\mathrm{NNI} - \text{median}(\mathrm{NNI_{random}})}{\text{median}(\mathrm{NNI_{uniform}}) - \text{median}(\mathrm{NNI_{random}})}
\end{equation}

\noindent \(\mathrm{NNI}\) is calculated for each sample image as
described above; \(\text{median}(\mathrm{NNI_{random}})\) and
\(\text{median}(\mathrm{NNI_{uniform}})\) are calculated from the
synthetic data specific to each sample image as described above. We
tested whether light treatment affects \(\mathrm{DI}\) and stomatla
density (\(D_S\)) using analysis of variance (ANOVA).

Finally, we examined the relationship between stomatal zone area and
stomatal length using a Bayesian linear mixed-effects model fit with the
\emph{R} package \textbf{brms} version 2.20.4
\citep{burkner_brms_2017, burkner_advanced_2018} and \emph{Stan} version
2.33.1 \citep{stan_development_team_stan_2023}. Stomatal zone area was
calculated using Voronoi tessellation
(e.g.~\autoref{fig:tessellation-example}). The stomatal zone area,
\(S_\text{area}\), is the region of the leaf surface whose distance to
stomate, \(S\), is less than the distance to any other stomate, \(S\).
Stomatal length was measured in ImageJ \citep{schneider_nih_2012}. We
modeled fixed effects of surface, light treatment, stomatal length, and
their 2- and 3-way interactions on \(\sqrt{S_\text{area}}\). We included
random intercepts, random effects of surface, random slopes, and random
surface-by-slope interactions within both plant and individual to
account for nonindependence of stomata within the same plant or
individual. We also modeled residual variance as a function of light
treatment. We sampled the posterior distribution from 4 chains with 1000
iterations each after 1000 warmup iterations. We calculated convergence
diagnostics (\(\hat{R}\)) and effective sample sizes following
\citet{vehtari_rank-normalization_2021}. We estimated the marginal slope
and 95\% highest posterior density (HPD) intervals between stomatal
length and \(\sqrt{S_\text{area}}\) using the \emph{emtrends} function
in the \emph{R} package \textbf{emmeans} version 1.8.8
\citep{lenth_emmeans_2023}.

\begin{figure}[ht]
\includegraphics[height = \textheight]{figures/tessellation-example.pdf}
\caption{Examples of synthetic and real leaf surfaces.  A) Uniform random synthetic leaf surface; B) Example of real leaf surface; C) Regularly distributed synthetic leaf surface. The zone defined by each stomate was calculated with voronoi tessellation and correlated with stomatal length in real leaves.}
\label{fig:tessellation-example}
\end{figure}

\hypertarget{paired-abaxial-and-adaxial-surface-analysis}{%
\subsection{Paired Abaxial and Adaxial Surface
Analysis}\label{paired-abaxial-and-adaxial-surface-analysis}}

To test whether the position of ab- and adaxial stomata are coordinated
we compared the observed distribution to a null distribution where the
positions on each surface are random. For each pair of surfaces
(observed or synthetic) we calculated the distance squared between each
pixel to the nearest stomatal centroid with the \emph{R} package
\textbf{raster} version 3.6.26. We refer to this as the `nearest
stomatal distance' or NSD. Then we calculated the pixel-wise Pearson
correlation coefficient. If stomatal positions on each surface are
coordinated to minimize the distance between mesophyll and the nearest
stomate, then we expect a negative correlation. A pixel that is far from
a stomate on one surface should be near a stomate on the other surface
(\autoref{fig:ideal-amphi-grid}). We generated a null distribution of
the correlation coefficient by simulating \(10^3\) synthetic data sets
for each observed pair. For each synthetic data set, we simulated
stomatal position using a random uniform distribution, as described
above, matching the number of stomata on abaxial and adaxial leaf
surfaces. Stomatal positions on each surface are coordinated if the
correlation coefficient is greater than 95\% of the synthetic
correlation values (one-tailed test).

\hypertarget{modeling-photosynthesis}{%
\subsection{Modeling Photosynthesis}\label{modeling-photosynthesis}}

We modeled photosynthesis CO\(_2\) assimilation rate using a
spatially-explicit two-dimensional reaction diffusion model using a
porous medium approximation \citep{parkhurst_diffusion_1994} using the
finite element method (FEM) following \citet{earles_excess_2017}.
Consider a two-dimensional leaf where stomata occur on each surface in a
regular sequence with interstomatal distance \(U\). The main outcome we
assessed is the advantage of offsetting the position of stomata on each
surface compared to have stomata on the same \(x\) position on each
surface. With these assumptions, by symmetry, we only need to model two
stomata, one abaxial and one adaxial, from \(x = 0\) to \(x = U/2\) and
from the adaxial surface at \(y = 0\) to the abaxial surface at
\(y = L\), the leaf thickness. We arbitrarily set the adaxial stomate at
\(x = 0\) and toggled the abaxial stomata position between \(x = U/2\)
(offset) or \(x = 0\) (below adaxial stomate). The `coordination
advantage' of offset stomatal position on each surface is the
photosynthetic rate of the leaf with offset stomata compared to that
with stomata aligned in the same \(x\) position:

\begin{equation} \label{eq:coordination_advantage}
  \text{coordination advantage} = \frac{A_\text{offset}}{A_\text{aligned}}
\end{equation}

We modeled the coordination advantage over a range of leaf thicknesses,
stomatal densities, photosynthetic capacities, and light environments to
understand when offsetting stomatal position on each surface might
deliver a significant photosynthetic advantage
(\autoref{tab:model_var}). The complete model description is available
in the Supporting Information.

\begin{table}

\caption{\label{tab:model_var}The parameter range of model variables tested for their effect on coordination advantage (\autoref{eq:coordination_advantage}) using a 2-D porous medium approximation. We used regularly spaced values within each range and simulated across all combinations. Here we converted model units to more conventional units (e.g. m to $\mu$m). $I_0$: PPFD incident on the leaf surface; $\varphi_\text{pal}$: Fraction of intercellular airspace (aka porosity), palisade; $T_\text{leaf}$: Leaf thickness; $U$: Interstomatal distance}
\centering
\begin{tabular}[t]{lll}
\toprule
Variable & Parameter range & Units\\
\midrule
$I_0$ & $50-1000$ & $\mu$mol m$^{-2}$ s$^{-1}$\\
$\varphi_\text{pal}$ & $0.1-0.3$ & m$^3$ airspace m$^{-3}$ leaf\\
$T_\text{leaf}$ & $101-501$ & $\mu$m\\
$U$ & $17-169$ & $\mu$m\\
\bottomrule
\end{tabular}
\end{table}

\hypertarget{results}{%
\section{Results}\label{results}}

Stomatal density of \emph{Arabidopsis thaliana} varies among light
treatments (ANOVA, \(F_{2,126} = 681, P = 2.88 \times 10^{-68}\))
because the density is much greater in the high light treatment
(\autoref{fig:density}). Density is consistently greater on abaxial leaf
surfaces across all light treatments (ANOVA,
\(F_{1,126} = 44.2, P = 8.21 \times 10^{-10}\); \autoref{fig:density}).
There is no evidence for an interaction between light treatment and
surface (ANOVA, \(F_{2,126} = 2.75 \times 10^{-2}, P = 0.973\)). Leaves
are amphistomatous with a mean stomatal density ratio of 0.44.

\begin{figure}[ht]
\includegraphics[width = 4in]{figures/density.pdf}
\caption{Stomatal density is higher in \textit{A. thaliana} plants grown under high light conditions. We determined statistical significance between light treatments using Tukey post-hoc tests. $^*~0.05 > P \ge 0.01; ^{**}~0.01 > P \ge 0.001; ^{***}~0.0001 > P \ge 0.0001; ^{***}~ P <0.0001$.}
\label{fig:density}
\end{figure}

\hypertarget{stomatal-distribution-is-nonrandom-but-far-from-ideal}{%
\subsection{Stomatal distribution is nonrandom, but far from
ideal}\label{stomatal-distribution-is-nonrandom-but-far-from-ideal}}

Many leaf surfaces (37 of 132, 28\%) are significantly overdispersed
compared to a random uniform distribution, but none were close to an
ideal hexagonal pattern (dispersion index = 1;
\autoref{fig:single-surface}). Before controlling for multiple
comparisons, 43.2\% are significantly overdispersed. The dispersion
index differs significantly among light treatments (ANOVA,
\(F_{2,126} = 8.55, P = 3.30 \times 10^{-4}\)) because the medium light
treatment is significantly less than the low treatment
(\autoref{fig:single-surface}). Dispersion index is consistently greater
on adaxial leaf surfaces across all light treatments (ANOVA,
\(F_{1,126} = 28.8, P = 3.67 \times 10^{-7}\);
\autoref{fig:single-surface}). There is no evidence for an interaction
between light treatment and surface (ANOVA,
\(F_{2,126} = 0.577, P = 0.563\)).

\begin{figure}[ht]
\includegraphics[width=4in]{figures/single-surface.pdf}
\caption{Stomata are more dispersed than expected under the null model of uniform random position (dispersion index = 0) but far from a distribution that maximizes distance between stomata (dispersion index = 1). We determined statistical significance between light treatments using Tukey post-hoc tests. $^*~0.05 > P \ge 0.01; ^{**}~0.01 > P \ge 0.001; ^{***}~0.0001 > P \ge 0.0001; ^{***}~ P <0.0001$.}
\label{fig:single-surface}
\end{figure}

\hypertarget{no-evidence-for-coordinated-stomatal-position-between-surfaces}{%
\subsection{No evidence for coordinated stomatal position between
surfaces}\label{no-evidence-for-coordinated-stomatal-position-between-surfaces}}

There is no evidence of spatial coordination between abaxial and adaxial
leaf surfaces. The pixel-wise correlation between nearest stomatal
distance (NSD) squared on paired abaxial and adaxial leaf surfaces is
not significantly less than zero in any of the 66 leaves
(\autoref{fig:dual-surface}). Before controlling for multiple
comparisons, 3\% are significantly \emph{positively} correlated. The NSD
correlation is not different among light treatments (ANOVA,
\(F_{2,63} = 2.28, P = 0.111\); \autoref{fig:dual-surface}).

\begin{figure}[ht]
\includegraphics[width = 4in]{figures/dual-surface.pdf}
\caption{Pixel-wise correlation between near stomatal distance (NSD) squared on paired abaxial and adaxial leaf surfaces. Dashed line indicates zero correlation. Weak positive correlations are not significantly different from zero after correcting for multiple comparisons. The correlation does not differ among light treatments.}
\label{fig:dual-surface}
\end{figure}

\hypertarget{larger-stomata-supply-larger-mesophyll-volumes}{%
\subsection{Larger stomata supply larger mesophyll
volumes}\label{larger-stomata-supply-larger-mesophyll-volumes}}

All parameters converged (\(\hat{R} < 1.01\)) and effective sample sizes
were exceeded \(10^3\). Across all light treatments and leaf surfaces,
stomatal length and stomatal area are weakly positively correlated
(\autoref{fig:length-area}). The slope was significantly greater than
zero for all abaxial surfaces, but not for the adaxial surface in low
and medium light treatments. The estimated marginal slopes and 95\% HPD
intervals for each combination of light and surface is: low light,
abaxial surface: 1.928 {[}\numrange{0.779}{3.133}{]}; low light, adaxial
surface: 1.745 {[}\numrange{-0.041}{3.373}{]}; medium light, abaxial
surface: 1.085 {[}\numrange{0.328}{1.957}{]}; medium light, adaxial
surface: 0.656 {[}\numrange{-0.399}{1.691}{]}; high light, abaxial
surface: 0.597 {[}\numrange{0.316}{0.911}{]}; high light, adaxial
surface: 1.269 {[}\numrange{0.831}{1.721}{]}.

\begin{figure}[ht]
\includegraphics[width = 4in]{figures/length-area.pdf}
\caption{Stomatal length and stomatal zone area are positively correlated. Linear regression lines and 95\% confidence ribbons are from a Bayesian linear mixed-effects model.}
\label{fig:length-area}
\end{figure}

\hypertarget{little-benefit-of-coordinated-stomatal-arrangement}{%
\subsection{Little benefit of coordinated stomatal
arrangement}\label{little-benefit-of-coordinated-stomatal-arrangement}}

We used the finite element method (FEM) to model CO\(_2\) diffusion
within the leaf and photosynthesis as a 2-D porous medium. Across all
realistic parts of parameter space, the coordination advantage is much
less than 0.01 (\autoref{fig:model_summary}). For reference, a
log-response of ratio is 0.01 is approximately 1\%. The only exception
was for thin leaves (\(T_\text{leaf} = 100~\mu \text{m}\)) with few
stomata (\(U = 338~\mu \text{m}\), which corresponds to a stomatal
density of \(\approx 10~\text{mm}^{-2}\)), where lateral diffusion is
major constraint on CO\(_2\) supply. However, such thin leaves with so
few stomata are uncommon among C\(_3\) plants (some CAM plants have low
stomatal density \citep{males_stomatal_2017}). In other areas of
parameter space, lateral diffusion limitations were small relative to
those along the ab-adaxial axis (see \autoref{fig:model_example} for a
representative model solution).

\begin{figure}[ht]
\includegraphics[width = 5in]{figures/model_summary.pdf}
\caption{There is little photosynthetic benefit of offsetting stomatal position each surface based on a 2-D model of photosynthesis. The coordination advantage (\autoref{eq:coordination_advantage}) is close to zero under nearly all of the parameter space \autoref{tab:model_var}, meaning that the photosynthetic rate of amphistomatous leaves with stomata optimally offset is nearly equal to leaves with stomata on each surface in the same position along the leaf plane. $I_0$: PPFD incident on the leaf surface; $\varphi_\text{pal}$: Fraction of intercellular airspace (aka porosity), palisade; $T_\text{leaf}$: Leaf thickness; $U$: Interstomatal distance.}
\label{fig:model_summary}
\end{figure}

\hypertarget{discussion}{%
\section{Discussion}\label{discussion}}

Stomata cost resources to maintain \citep{deans_optimization_2020} and
expose leaves to risks such as hydraulic failure
\citep{wang_theoretical_2020} or infection by plant pathogens
\citep{melotto_stomatal_2017}. Therefore leaves should develop enough
stomata to adequately supply CO\(_2\) to chloroplasts, but not
overinvest. A widespread hypothesis in plant ecophysiology is that
natural selection optimizes traits like stomatal size, density, and
distribution to maximize carbon gain relative to any costs in a given
environmental context. In principle, spacing stomata to minimize the
average distance between stomatal pores and chloroplasts within the
mesophyll should increase carbon gain, all else being equal. However,
reducing this distance to its absolute minimum may be constrained by
developmental processes, or the photosynthetic benefit may be too small
to be `seen' by natural selection (i.e.~the selection coefficient is
less than drift barrier \citep{sung_drift-barrier_2012}).

We tested five related hypotheses about stomatal spacing in
amphistomatous leaves using the model angiosperm \emph{Arabidopsis
thaliana} grown under different light intensities. First, we predicted
that stomata on each surface are overdispersed relative to a random
uniform distribution, which should increase CO\(_2\) supply. Stomata on
each surface are overdispersed (\autoref{fig:single-surface}), but are
not ideally dispersed in an equilateral triangular grid as would be
optimal to minimize CO\(_2\) diffusion path length and equalize the area
supplied by each stomate (\autoref{fig:tessellation-example}). Second,
we predicted that an optimal amphistomatous leaf has offset stomata such
that stomata are more likely to appear on one leaf surface if there is
not a stomata directly opposite it on the other surface as shown in
\autoref{fig:ideal-amphi-grid}. However, there is no evidence for
coordination and the positions on each surface appear independent,
regardless of light treatment (\autoref{fig:dual-surface}). Third, we
predicted that plants respond plastically to higher light intensity by
increasing stomatal density. \emph{Arabidopsis} plants grown under high
light had higher stomatal density than the same genotype growns under
low and medium light intensity (\autoref{fig:density}). However, we
found no support for our fourth prediction that stomata would be more
evenly dispersed at high light intensity (\autoref{fig:single-surface}).
Finally, we predicted that within leaf variation in stomatal size would
correlate with stomatal spacing, as larger stomata can supply larger
volumes of adjacent mesophyll. In all three light treatments, stomatal
size positively covaried with the stomatal zone, i.e.~adjacent region of
mesophyll that would be supplied by that stomate
(\autoref{fig:length-area}).

Stomatal spacing on \emph{A. thaliana} leaves partially supports our
overall hypothesis that natural selection minimizes the average distance
between stomata and chloroplasts, for a given overall stomatal density.
There are three nonmutually exclusive hypotheses for why several of our
predictions were wrong. First, our predictions are wrong becuase they
are based on overly simplistic assumptions about epidermal and mesophyll
anatomy. Second, natural selection may be constrained by developmental
processes that prevent phenotypes from reaching their adaptive optima.
Third, the benefit of some traits may be of too little consequence to
result in fitness differences large enough to be respond to selection.
We consider the plausibility of these alternative hypotheses below and
present ideas for future work to test them.

{[}this paragraph should discuss why our assumptions may be too
simplistic{]} We assume an idealized leaf epidermal and mesophyll
structure that is homogenous and unconstrained by other tradeoffs. Real
leaves not only provide pathways for CO\(_2\) diffusion, but must supply
water, intercept light, and deter herbivores and pathogens. These
competing interests results in nonuniform epidermal and mesophyll
structure that could alter our predictions about optimal stomatal
spacing. {[}explain some examples{]}

{[}this paragraph should discuss developmental constraint{]} No
developmental pathway exists to ensure the ideal placement of stomata on
the leaf. {[}add more explanation{]}

{[}this paragraph should discuss limits on natural selection{]} The gas
exchange model demonstrate little photosynthetic gain from
abaxial-adaxial stomatal coordination (\autoref{fig:model_summary}).
Even though lateral diffusion may limit photosynthesis
\citep{morison_lateral_2005}, the marginal gain from optimally
offsetting stomata is not sufficient to generate fitness differences
relative to the strength of genetic drift (i.e.~the drift-barrier). We
can similarly extrapolate that an ideal, equilateral triangular stomatal
spacing is only slightly better than a suboptimal overdispered pattern.
Explaining these observations as the result of weak selection is in
tension with the finding that stomatal size and zone positively covary,
which would suggest that small changes in lateral diffusion distance are
significant. As described above, the positive correlation between
stomatal size and zone may be explained by common developmental
processes rather than as an adaptation to maximize CO\(_2\) diffusion.

{[}discuss significance of our study and future directions{]} Our study
corroborates previous studies which demonstrate that stomata are
non-randomly distributed along the leaf surface as a result of
developmental mechanisms such as spatially biased arrest of stomatal
initials \citep{boetsch_arrest_1995}, oriented asymmetric cell division
\citep{geisler_oriented_2000}, and cell cycle controls
\citep{croxdale_stomatal_2000}. We do not investigate the potential
developmental pathways that influence stomatal dispersion in this study;
however, they are important to consider as these pathways could limit
plants from reaching the theoretical peak in the adaptive landscape:
uniform stomatal dispersion. Instead, as this study suggests, plants may
simply compensate with higher stomatal density and by fitting stomatal
size to the area that they supply with CO\(_2\). To understand why
stomata are not ideally dispersed, more modelling should be done to
estimate the fitness gain of stomatal dispersion. Additionally, genetic
manipulation studies should attempt to create mutants with clustered and
ideally dispersed stomata for a comparison of their photosynthetic
traits. This could have extremely important implications for maximum
assimilation rates in crops as most crop species are grown in high light
where CO\(_2\) is often limiting. In drought-prone environments,
increased stomatal dispersion may increase water use efficiency by
reducing the number of stomata needed to achieve the same internal
CO\(_2\) concentration, \(C_\text{i}\).

{[}Conclusion{]} Our results suggest that after optimiziing stomatal
density and having developmental rules for spacing stomata relatively
evenly, there may be limited gains to further optimization. Therefore,
developmental constraints may be necessary to make sense of some
features of stomatal spacing and distribution.

\hypertarget{references}{%
\section{References}\label{references}}

\section{Competing interests}

The authors declare no competing interests.

\section{Author contributions statement}

JLW and CDM conceived of the project, analyzed data, and wrote the
manuscript. GW provided data. TNB contributed to model development and
helped edit the manuscript.



\bibliographystyle{abbrvnat}
\bibliography{stomata-spacing.bib}

%% Author bio-pics with images


\end{document}
