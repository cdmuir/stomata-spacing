
\documentclass[12pt,halfline,a4paper,]{ouparticle}

% Packages I think are necessary for basic Rmarkdown functionality
\usepackage{hyperref}
\usepackage{graphicx}
\usepackage{listings}
\usepackage{xcolor}
\usepackage{fancyvrb}
\usepackage{framed}

% Link coloring
\hypersetup{breaklinks=true,
            bookmarks=true,
            pdfauthor={},
            pdftitle={Non-independence of Leaf Surfaces in Amphistomatous Leaves of Arabidopsis thaliana and a 2D Model for Photosynthesis}
            }


% For knitr::kable functionality
\usepackage{booktabs}
\usepackage{longtable}

%% To allow better options for figure placement
%\usepackage{float}

% Packages that are supposedly required by OUP sty file
\usepackage{amssymb, amsmath, geometry, amsfonts, verbatim, endnotes, setspace}

% For code highlighting I think
\DefineVerbatimEnvironment{Highlighting}{Verbatim}{commandchars=\\\{\}}
\definecolor{shadecolor}{RGB}{248,248,248}
\newenvironment{Shaded}{\begin{snugshade}}{\end{snugshade}}
\newcommand{\AlertTok}[1]{\textcolor[rgb]{0.94,0.16,0.16}{#1}}
\newcommand{\AnnotationTok}[1]{\textcolor[rgb]{0.56,0.35,0.01}{\textbf{\textit{#1}}}}
\newcommand{\AttributeTok}[1]{\textcolor[rgb]{0.77,0.63,0.00}{#1}}
\newcommand{\BaseNTok}[1]{\textcolor[rgb]{0.00,0.00,0.81}{#1}}
\newcommand{\BuiltInTok}[1]{#1}
\newcommand{\CharTok}[1]{\textcolor[rgb]{0.31,0.60,0.02}{#1}}
\newcommand{\CommentTok}[1]{\textcolor[rgb]{0.56,0.35,0.01}{\textit{#1}}}
\newcommand{\CommentVarTok}[1]{\textcolor[rgb]{0.56,0.35,0.01}{\textbf{\textit{#1}}}}
\newcommand{\ConstantTok}[1]{\textcolor[rgb]{0.00,0.00,0.00}{#1}}
\newcommand{\ControlFlowTok}[1]{\textcolor[rgb]{0.13,0.29,0.53}{\textbf{#1}}}
\newcommand{\DataTypeTok}[1]{\textcolor[rgb]{0.13,0.29,0.53}{#1}}
\newcommand{\DecValTok}[1]{\textcolor[rgb]{0.00,0.00,0.81}{#1}}
\newcommand{\DocumentationTok}[1]{\textcolor[rgb]{0.56,0.35,0.01}{\textbf{\textit{#1}}}}
\newcommand{\ErrorTok}[1]{\textcolor[rgb]{0.64,0.00,0.00}{\textbf{#1}}}
\newcommand{\ExtensionTok}[1]{#1}
\newcommand{\FloatTok}[1]{\textcolor[rgb]{0.00,0.00,0.81}{#1}}
\newcommand{\FunctionTok}[1]{\textcolor[rgb]{0.00,0.00,0.00}{#1}}
\newcommand{\ImportTok}[1]{#1}
\newcommand{\InformationTok}[1]{\textcolor[rgb]{0.56,0.35,0.01}{\textbf{\textit{#1}}}}
\newcommand{\KeywordTok}[1]{\textcolor[rgb]{0.13,0.29,0.53}{\textbf{#1}}}
\newcommand{\NormalTok}[1]{#1}
\newcommand{\OperatorTok}[1]{\textcolor[rgb]{0.81,0.36,0.00}{\textbf{#1}}}
\newcommand{\OtherTok}[1]{\textcolor[rgb]{0.56,0.35,0.01}{#1}}
\newcommand{\PreprocessorTok}[1]{\textcolor[rgb]{0.56,0.35,0.01}{\textit{#1}}}
\newcommand{\RegionMarkerTok}[1]{#1}
\newcommand{\SpecialCharTok}[1]{\textcolor[rgb]{0.00,0.00,0.00}{#1}}
\newcommand{\SpecialStringTok}[1]{\textcolor[rgb]{0.31,0.60,0.02}{#1}}
\newcommand{\StringTok}[1]{\textcolor[rgb]{0.31,0.60,0.02}{#1}}
\newcommand{\VariableTok}[1]{\textcolor[rgb]{0.00,0.00,0.00}{#1}}
\newcommand{\VerbatimStringTok}[1]{\textcolor[rgb]{0.31,0.60,0.02}{#1}}
\newcommand{\WarningTok}[1]{\textcolor[rgb]{0.56,0.35,0.01}{\textbf{\textit{#1}}}}

% use upquote if available, for straight quotes in verbatim environments
\IfFileExists{upquote.sty}{\usepackage{upquote}}{}

% For making Rmarkdown lists
\providecommand{\tightlist}{%
  \setlength{\itemsep}{0pt}\setlength{\parskip}{0pt}}

% Macros for dealing with affiliations, footnotes, etc.
\makeatletter
\def\Newlabel#1#2#3{\expandafter\gdef\csname #1@#2\endcsname{#3}}

\def\Ref#1#2{\@ifundefined{#1@#2}{???}{\csname #1@#2\endcsname}}

\newcommand*\samethanks[1][\value{footnote}]{\footnotemark[#1]}

\newcommand*\ifcounter[1]{%
  \ifcsname c@#1\endcsname
    \expandafter\@firstoftwo
  \else
    \expandafter\@secondoftwo
  \fi
}

\newcommand*\thanksbycode[1]{%
  \ifcounter{FNCT@#1}
    {\samethanks[\value{FNCT@#1}]}
    {\thanks{\Ref{FN}{#1}}\newcounter{FNCT@#1}\setcounter{FNCT@#1}{\value{footnote}}}
}

% Create labels for Addresses if the are given in Elsevier format
\Newlabel{ADR}{Colgate University}{13 Oak Drive, Hamilton, NY 13346}
\Newlabel{ADR}{University of Hawaii at Manoa}{2500 Campus Rd, Honolulu,
HI 96822}

% Create labels for Footnotes if the are given in Elsevier format
\Newlabel{FN}{1}{Equal contribution}
\Newlabel{FN}{2}{Current email address:
\href{mailto:cat@example.com}{cat@example.com}}

% Part for setting citation format package: natbib

% Part for setting citation format package: biblatex

% Part for indenting CSL refs
% Pandoc citation processing
\newlength{\csllabelwidth}
\setlength{\csllabelwidth}{3em}
\newlength{\cslhangindent}
\setlength{\cslhangindent}{1.5em}
% for Pandoc 2.8 to 2.10.1
\newenvironment{cslreferences}%
  {}%
  {\par}
% For Pandoc 2.11+
\newenvironment{CSLReferences}[2] % #1 hanging-ident, #2 entry spacing
 {% don't indent paragraphs
  \setlength{\parindent}{0pt}
  % turn on hanging indent if param 1 is 1
  \ifodd #1 \everypar{\setlength{\hangindent}{\cslhangindent}}\ignorespaces\fi
  % set entry spacing
  \ifnum #2 > 0
  \setlength{\parskip}{#2\baselineskip}
  \fi
 }%
 {}
\usepackage{calc} % for calculating minipage widths
\newcommand{\CSLBlock}[1]{#1\hfill\break}
\newcommand{\CSLLeftMargin}[1]{\parbox[t]{\csllabelwidth}{#1}}
\newcommand{\CSLRightInline}[1]{\parbox[t]{\linewidth - \csllabelwidth}{#1}\break}
\newcommand{\CSLIndent}[1]{\hspace{\cslhangindent}#1}
% Pandoc header
\usepackage[nomarkers,tablesfirst]{endfloat}
\usepackage{lineno}
\linenumbers

\begin{document}

\title{Non-independence of Leaf Surfaces in Amphistomatous Leaves of
\emph{Arabidopsis thaliana} and a 2D Model for Photosynthesis}

\author{%
%
% Code for old style authors field
%
% Add \and if both authors and author
%
%
% Code for new (elsevier) style author field
\name{Jacob L. Watts}
\address{\Ref{ADR}{Colgate University}}
%
\email{\href{mailto:jwatts@colgate.edu}{jwatts@colgate.edu}}%
\thanks{Corresponding author; Email: \href{mailto:jwatts@colgate.edu}{jwatts@colgate.edu}}%
%
%
\and
\name{Graham Dow}
\address{\Ref{ADR}{ETH}}
%
\email{\href{mailto:graham.dow@usys.ethz.ch}{graham.dow@usys.ethz.ch}}%
%
%
%
\and
\name{Thomas N. Buckley}
\address{\Ref{ADR}{University of California Davis}}
%
\email{\href{mailto:tnbuckley@ucdavis.edu}{tnbuckley@ucdavis.edu}}%
%
%
\thanksbycode{1}
%
\and
\name{Christopher D. Muir}
\address{\Ref{ADR}{University of Hawaii at Manoa}}
%
\email{\href{mailto:cdmuir@hawaii.edu}{cdmuir@hawaii.edu}}%
%
%
\thanksbycode{1}
%
%
}

\abstract{This is the abstract.

It consists of two paragraphs.}

\date{\today}

\keywords{stomatal arrangement; Aribidopsis thaliana; amphistomatic
leaves; modeling photosynthesis; optimizing photosynthesis}

\maketitle



\hypertarget{introduction}{%
\section{Introduction}\label{introduction}}

Stomatal anatomy (e.g.~size, density, distribution, and patterning) and
physiological responses to the environment regulate gas exchange of
CO\(_2\) assimilation and water loss through transpiration. As waxy
cuticles are mostly impermeable to CO\(_2\) and \(H_2O\), stomata are
discrete points accounting for a small percentage of the leaf area
through which gas exchange occurs
(\protect\hyperlink{ref-lange_responses_1971}{Lange et al. 1971}). They
consist of two guard cells which open and close upon changes in turgor
pressure or hormonal cues
(\protect\hyperlink{ref-mcadam_linking_2016}{McAdam and Brodribb 2016}).
The stomatal pore leads to an internal space known as the substomatal
cavity where gases (and pathogens) contact the mesophyll
(\protect\hyperlink{ref-muir_stomatal_2020}{Christopher D. Muir 2020}).
Once in the mesophyll, CO\(_2\) diffuses throughout a network of
intercellular air space (IAS) and into mesophyll cells where CO\(_2\)
assimilation (\(A\)) occurs within the chloroplasts
(\protect\hyperlink{ref-lee_diffusion_1964}{Lee and Gates 1964}).
Maximum conductance (\(g_\text{max}\)) and transpiration are determined
by numerous environmental and anatomical parameters such as: vapor
pressure deficit (VPD), irradiance, temperature, wind speed, leaf water
potential, IAS geometry, mesophyll cell anatomy, and stomatal anatomy.

Natural selection should optimize the anatomy and physiology of stomata
to maximize CO\(_2\) gain per unit of water loss, thereby maximizing
assimilation rates for a given set of hydraulic constraints
(\protect\hyperlink{ref-cowan_stomatal_1977}{Cowan and Farquhar 1977};
\protect\hyperlink{ref-buckley_optimal_2017}{Buckley, Sack, and Farquhar
2017}; \protect\hyperlink{ref-sperry_predicting_2017}{Sperry et al.
2017}). However, optimal stomatal anatomy and physiology depends on
developmental and environmental constraints and other physiological
co-variates {[}\protect\hyperlink{ref-croxdale_stomatal_2000}{Croxdale}
(\protect\hyperlink{ref-croxdale_stomatal_2000}{2000});
\protect\hyperlink{ref-harrison_influence_2020}{Harrison et al.}
(\protect\hyperlink{ref-harrison_influence_2020}{2020}); MUIR IN
PRESS{]}. There is wide variety of stomatal traits in nature, but
certain common features as well. For example, almost all stomata follow
the one cell spacing rule to maintain proper stomatal functioning
(\protect\hyperlink{ref-geisler_oriented_2000}{Geisler, Nadeau, and Sack
2000}; \protect\hyperlink{ref-dow_physiological_2014}{Dow, Berry, and
Bergmann 2014}); however some species (notably in \emph{Begonia}) appear
to benefit from overlapping vapor shells caused by stomatal clustering
(\protect\hyperlink{ref-yi_gan_stomatal_2010}{Yi Gan et al. 2010};
\protect\hyperlink{ref-lehmann_effects_2015}{Lehmann and Or 2015};
\protect\hyperlink{ref-papanatsiou_stomatal_2017}{Papanatsiou, Amtmann,
and Blatt 2017}). Stomatal density positively co-varies with irradiance
during leaf development and negatively co-varies with CO\(_2\)
concentration (\protect\hyperlink{ref-gay_influence_1975}{Gay and Hurd
1975}; \protect\hyperlink{ref-schoch_dependence_1980}{Schoch, Zinsou,
and Sibi 1980}; \protect\hyperlink{ref-woodward_stomatal_1987}{Woodward
1987}; \protect\hyperlink{ref-royer_stomatal_2001}{Royer 2001}).
Stomatal size is jointly controlled by genome size, light, and stomatal
density (\protect\hyperlink{ref-jordan_environmental_2015}{Jordan et al.
2015}). Size positively co-varies with genome size
(\protect\hyperlink{ref-roddy_scaling_2020}{Roddy et al. 2020}) and
negatively co-varies with stomatal density
(\protect\hyperlink{ref-camargo_density_2011}{Camargo and Marenco
2011}). Total stomatal area (size * density) is optimized for
operational conductance (\(gs_\text{op}\)) rather than maximum
conductance (\(gs_\text{max}\)) such that stomatal apertures are most
responsive to changes in the environment at operational aperture
(\protect\hyperlink{ref-franks_physiological_2012}{Franks et al. 2012};
\protect\hyperlink{ref-liu_scaling_2021}{Liu et al. 2021}). Stomatal
aperture can compensate for maladaptive stomatal densities to an extent
(\protect\hyperlink{ref-bussis_stomatal_2006}{Büssis et al. 2006}), but
stomatal density and size ultimately determine a leaf's theoretical
\(gs_\text{max}\) (\protect\hyperlink{ref-sack_developmental_2016}{Sack
and Buckley 2016}). Low stomatal densities lead to irregular and
insufficient CO\(_2\) supply and reduced photosynthetic efficiency in
areas with no stomata
(\protect\hyperlink{ref-pieruschka_lateral_2006}{Roland Pieruschka et
al. 2006}; \protect\hyperlink{ref-morison_lateral_2005}{Morison et al.
2005}), while high stomatal densities can reduce water use efficiency
(WUE) (\protect\hyperlink{ref-bussis_stomatal_2006}{Büssis et al. 2006})
and incur excessive metabolic costs
(\protect\hyperlink{ref-deans_optimization_2020}{Deans et al. 2020}). In
most species, stomata occur on the abaxial (usually lower) leaf surface;
but amphistomy, the occurrence of stomata on both abaxial and adaxial
leaf surfaces, is also prevalent in high light environments with
constant or intermittent access to sufficient water
(\protect\hyperlink{ref-mott_adaptive_1982}{Mott, Gibson, and O'Leary
1982}; \protect\hyperlink{ref-jordan_using_2014}{Jordan, Carpenter, and
Brodribb 2014}; \protect\hyperlink{ref-muir_light_2018}{Christopher D.
Muir 2018}; \protect\hyperlink{ref-drake_two_2019}{Drake et al. 2019};
\protect\hyperlink{ref-muir_is_2019}{Christopher D. Muir 2019}).
Amphistomy effectively halves CO\(_2\) diffusion path length and
boundary layer resistance by doubling boundary layer area
(\protect\hyperlink{ref-parkhurst_adaptive_1978}{Parkhurst 1978};
\protect\hyperlink{ref-harrison_influence_2020}{Harrison et al. 2020};
\protect\hyperlink{ref-mott_amphistomy_1991}{Mott and Michaelson 1991}).
Historically, stomatal patterning in dicot angiosperms was thought to be
random with an exclusionary distance surrounding each stomate
(\protect\hyperlink{ref-sachs_developmental_1974}{Sachs 1974}); however,
the developmental controls of stomatal patterning are poorly understood
and likely more complex than random development along the leaf surface.
\protect\hyperlink{ref-croxdale_stomatal_2000}{Croxdale}
(\protect\hyperlink{ref-croxdale_stomatal_2000}{2000}){]} reviews three
developmental theories which attempt to explain stomatal patterning in
angiosperms: inhibition, cell lineage, and cell cycle, ultimately
arguing for a cell cycle based control of stomatal patterning.

The patterning and spacing of stomata on the leaf affects photosynthesis
in \(C_3\) leaves by altering the CO\(_2\) diffusion path length from
stomata to sites of carboxylation in the mesophyll. Maximum
photosynthetic rate (\(A_\text{max}\)) in \(C_3\) plants is limited by
biochemistry and diffusion
(\protect\hyperlink{ref-parkhurst_intercellular_1990}{Parkhurst and Mott
1990}; \protect\hyperlink{ref-manter_ci_2004}{Manter 2004};
\protect\hyperlink{ref-carriqui_diffusional_2015}{Carriquí et al.
2015}). In low light environments, assimilation rates are low;
consequently, CO\(_2\) flux is low and the internal CO\(_2\)
concentration (\(C_\text{i}\)) remains high; thus \(A_\text{max}\) is
constrained by biochemical limitations in low light
(\protect\hyperlink{ref-kaiser_metabolic_2016}{Kaiser et al. 2016}). For
well hydrated leaves in high light, photosynthesis is often limited by
CO\(_2\) supply as resistances from the boundary layer, stomatal pore,
and mesophyll can result in insufficient \(C_\text{i}\) to supply
efficient photosynthesis at the chloroplast
{[}\protect\hyperlink{ref-farquhar_biochemical_1980}{Farquhar,
Caemmerer, and Berry}
(\protect\hyperlink{ref-farquhar_biochemical_1980}{1980});
lehmeier\_cell\_2017{]}. In this study, we focus primarily on how
stomatal patterning affects diffusion. To maximize CO\(_2\) supply from
the stomatal pore to chloroplast, stomata should be uniformly
distributed in an equilateral triangular grid on the leaf surface so as
to minimize stomatal number and CO\(_2\) diffusion path length. As the
diffusion rate of CO\(_2\) though liquid is approximately 10,000 times
slower than CO\(_2\) diffusion through air, mesophyll resistance is
generally thought to be primarily limited by liquid diffusion
(\protect\hyperlink{ref-aalto_three-dimensional_2002}{Aalto and Juurola
2002}), but diffusion through the IAS has also been shown to be a rate
limiting process because of the tortuous, disjunct nature of IAS
(\protect\hyperlink{ref-harwood_understanding_2021}{Harwood,
Théroux‐Rancourt, and Barbour 2021}). Additionally, tortuosity is higher
in horizontal directions (parallel to leaf surface) than vertical
directions (perpendicular to leaf surface) because of the cylindrical
shape and vertical arrangement of pallisade mesophyll cells
(\protect\hyperlink{ref-earles_beyond_2018}{Earles et al. 2018};
\protect\hyperlink{ref-harwood_understanding_2021}{Harwood,
Théroux‐Rancourt, and Barbour 2021}); however, the ratio of lateral to
vertical diffusion rate is still largely unknown
(\protect\hyperlink{ref-morison_lateral_2005}{Morison et al. 2005};
\protect\hyperlink{ref-pieruschka_lateral_2005}{R. Pieruschka 2005};
\protect\hyperlink{ref-pieruschka_lateral_2006}{Roland Pieruschka et al.
2006}). Depending on the thickness of the leaf, porosity of the leaf
mesophyll, tortuosity of the IAS, and lateral to vertical diffusion rate
ratio, minimizing diffusion path length for CO\(_2\) may yield
significant increases in CO\(_2\) supply for photosynthesis and higher
\(A_\text{max}\).

We hypothesized natural selection will favor stomatal patterning and
distribution to minimize the diffusion path length. In amphistomatous
leaves, this would be accomplished by 1) a uniform distribution of
stomata on both abaxial and adaxial leaf surfaces and 2) coordinated
stomatal spacing on each surface that offsets the position of stomata
(Fig. \ref{fig:amphi_grid}). Coordination between leaf surfaces is
defined in this study as the occurrence of stomata in areas farther from
stomata on the opposite leaf surface. Additionally, because CO\(_2\) is
more limiting for photosynthesis under high light, we hypothesize that
in high light 3) there should be more stomata, and 4) stomata should be
more uniformly distributed than in low light. Finally, as stomatal
densities are selected for optimal operational aperture, we hypothesize
that 5) stomatal length will be positively correlated with the area of
the leaf surface to which it is closest (stomatal zone area: the area a
stomate occupies and supplies with CO\(_2\)). This way, each stomate can
be optimally sized for its relative leaf area and associated CO\(_2\)
demand.

To test these hypotheses, we grew \emph{A. thaliana} in high, medium,
and low light and measured stomatal density, size, and patterning on
both leaf surfaces, and spatial coordination between them. We use
voronoi tessellation techniques to calculate the stomatal leaf area. We
also use CO\(_2\) diffusion modeling to ask: what environmental and
physiological conditions maximize the photosynthetic advantage
(\(A\text{gain}\)) of abaxial-adaxial stomatal coordination?
Specifically, we predicted that traits which affect diffusion path
length (leaf thickness, stomatal density, leaf porosity,
lateral-vertical diffusion rate ratio), diffusion rate (temperature,
pressure), and CO\(_2\) demand (Rubisco concentration, light) would
modulate the advantage of optimal stomatal arrangement following the
relationships outlined in Table \ref{tab:hypotheses}. Here, we integrate
over reasonable parameter space to determine the ecophysiological
context most likely to favor stomatal spatial coordination in
amphistomatous leaves.

\begin{table}[ht]
\centering
\begin{tabular}{rll}
  \hline
 & trait & relationship \\ 
  \hline
1 & leaf thickness & + \\ 
  2 & stomatal density & - \\ 
  3 & leaf porosity & - \\ 
  4 & lat.-vert. diffusion ratio & - \\ 
  5 & temperature & - \\ 
  6 & pressure & - \\ 
  7 & Rubisco concentration & + \\ 
  8 & light & + \\ 
   \hline
\end{tabular}
\caption{A summary of the hypothesized relationships between leaf traits and environmental conditions and photosynthetic advantage of stomatal spatial coordination in amphistomatous leaves.} 
\label{tab:hypotheses}
\end{table}

\hypertarget{materials-and-methods}{%
\section{Materials and methods}\label{materials-and-methods}}

\hypertarget{data-preparation}{%
\subsection{Data Preparation}\label{data-preparation}}

{[}CDM: Graham Dow provided these images. We'll need to add him as a
co-author and ask him to write methods on image acquisition.{]}

\emph{Arabidopsis thaliana} plants were grown in three different light
environments: low light (50 PAR), medium light (100 PAR), and high light
(200 PAR) following the light standards from (CITE). Once leaves were
mature, we captured images of the abaxial and adaxial leaf surfaces
using xxx bifocal microscope. This microscope allows the capture of two
focal lengths that are spatially correlated such that pixel (1,1) of the
abaxial surface image is directly above pixel (1,1) of the adaxial
surface image, and so on\ldots{} Images were 512 by 512 pixels and
covered X leaf surface area. We captured 132 images in total, making 66
abaxial-adaxial image pairs. The position of all stomata were recorded
using ImageJ
(\protect\hyperlink{ref-schneider_nih_2012}{\textbf{schneider\_nih\_2012?}}).

\hypertarget{single-surface-analysis}{%
\subsection{Single Surface Analysis}\label{single-surface-analysis}}

We tested whether stomata are non-randomly distributed by comparing the
observed stomatal patterning to a random uniform pattern. For each leaf
surface image with \(n\) stomata we generated \(10^3\) synthetic
surfaces with \(n\) stomata uniformly randomly distributed on the
surface. For each sample image, we compared the observed Nearest
Neighbor Index (\(\mathrm{NNI}\)) to the null distribution of
\(\mathrm{NNI}\) values calculated from the synthetic data set. The
observed stomatal distribution is dispersed relative to a uniform random
distribution if the observed \(NNI\) is greater than 95\% synthetic
\(\mathrm{NNI}\) values (one-tailed test).

\(\mathrm{NNI}\) is the ratio of observed mean distance
(\(\overline{D}_O\)) to the expected mean distance (\(\overline{D}_E\))
where \(\overline{D}_E\) is:

\begin{equation}\label{eq:emd}
  \mathrm{\overline{D}_E} = \frac{0.5}{\sqrt{A_l/n_\text{stomata}}}
\end{equation}

\(A_l\) is leaf area and \(n_\text{stomata}\) the number of stomata.
\(\overline{D}_E\) is the theoretical average distance to the nearest
neighbor of each stomate if stomata were randomly distributed (CITE;
Clark and Evans, 1954). And \(\overline{D}_O\) calculated for each
synthetic data set is:

\begin{equation}\label{eq:omd}
  \mathrm{\overline{D}_O} = \frac{\sum_{i=1}^{n_\text{stomata}}d_i}{n_\text{stomata}}
\end{equation}

\(d_i\) is the distance between \(\text{stomate}_i\) and its nearest
neighbor. We calculated \(\mathrm{NNI}\) using the R package
\textbf{spatialEco} version 1.3.7.

For each sample image, we also simulated \(10^3\) synthetic data with
\(n\) stomata ideally dispersed in an equilateral triangular grid. For
these grids, we integrated over plausible stomatal densities and then
conditioned on stomatal grids with exactly \(n\) stomata. The simulated
stomatal count was drawn from a Poisson distribution with the mean
parameter \(\lambda\) drawn from a Gamma distribution with shape \(n\)
and scale 1 \(\lambda \sim \Gamma(n, 1)\). \(\Gamma(n, 1)\) is the
posterior distribution of \(\lambda\) with a flat prior distribution.
This allows us to integrate over uncertainty in the stomatal density
from the sample image.

We calculated the dispersion index \(\mathrm{DI}\)) which varies from
zero to one. Zero is uniformly random and one is ideally dispersed:

\begin{equation}\label{eq:disp}
  \mathrm{DI} = \frac{\mathrm{NNI} - \text{median}(\mathrm{NNI_{random}})}{\text{median}(\mathrm{NNI_{uniform}}) - \text{median}(\mathrm{NNI_{random}})}
\end{equation}

Uniformly distributed stomata on a leaf with stomatal density (\(D_S\))
maximize \(\overline{D}_O\) in an equilateral triangle pattern with side
lengths (\(s\)) equal to two times the incircle radius or apothem,
\(a\), of a regular hexagon with area (\(A_\text{hex}\)) equal to the
area of the leaf (\(A_l\)) divided by the number of stomata
(\(n_\text{stomata}\)) on the leaf according to Eq. \ref{eq:eq2}:

\begin{equation}\label{eq:eq2}
  s = \frac{\sqrt{2}} {3^{1/4} \sqrt{n_\text{stomata} / A_l}}
\end{equation}

As all sides of an equilateral triangle have the same length, the
average nearest neighbor distance for a given \(D_S\) is maximized and
each stomata occupies the same hexagonal area (\(A_\text{hex}\)). The
closer to one \(DI\) is, the more uniform an area each of its stomata
supply with CO\(_2\) during photosynthesis. We tested whether light
treatment affects \(\mathrm{DI}\) and \(D_S\) using ANOVAs.

Finally, to test our hypothesis that stomatal length is modulated by the
area of the leaf to which it supplies CO\(_2\), we examined the
relationship between stomatal zone area and stomatal length using a
Bayesian generalized non-linear multilevel model with the R package
\textbf{brms} version 2.15.0. Stomatal zone area was calculated using
voronoi tessellation (e.g.~Fig. \ref{fig:example}). The stomatal zone
area, \(S_\text{area}\), is the region of the leaf surface whose
distance to stomate, \(S\), is less than the distance to any other
stomate, \(S\). Stomatal length was measured in \texttt{ImageJ}
(\protect\hyperlink{ref-schneider_nih_2012}{\textbf{schneider\_nih\_2012?}}).

\hypertarget{paired-abaxial-and-adaxial-surface-analysis}{%
\subsection{Paired Abaxial and Adaxial Surface
Analysis}\label{paired-abaxial-and-adaxial-surface-analysis}}

To test whether the position of ab- and adaxial stomata are coordinated
we compared the observed distribution to a null distribution where the
positions on each surface are random. For each pair of surfaces
(observed or synthetic) we calculated the distance squared between each
to the nearest stomatal centroid with the R package \textbf{raster}
version 3.4.13. Then we calculated the cell-wise Pearson correlation
coefficient. If stomatal positions on each surface are coordinated to
minimize the distance between mesophyll and the nearest stomate then we
expect a negative correlation. A cell that is far from a stomate on one
surface should be near a stomate on the other surface (Fig.
<<<<<<< HEAD
\ref{fig:amphi_grid}). We generated a null distribution of the
correlation coefficient by simulating \(10^3\) synthetic data sets for
each observed pair. For each synthetic data set, we simulated stomatal
position using a random uniform distribution, as described above,
matching the number of stomata on abaxial and adaxial leaf surfaces.
Stomatal positions on each surface are coordinated if the correlation
coefficient is greater greater than 95\% of the synthetic correlation
values (one-tailed test).
=======
\ref{fig:amphi}). We generated a null distribution of the correlation
coefficient by simulating \(10^3\) synthetic data sets for each observed
pair. For each synthetic data set, we simulated stomatal position using
a random uniform distribution, as described above, matching the number
of stomata on abaxial and adaxial leaf surfaces. Stomatal positions on
each surface are coordinated if the correlation coefficient is greater
greater than 95\% of the synthetic correlation values (one-tailed test).
>>>>>>> a2d63857818f934454237c3ab5c736564b8f8ef9

\hypertarget{modeling-photosynthesis}{%
\subsection{Modeling Photosynthesis}\label{modeling-photosynthesis}}

We modeled photosynthesis CO\(_2\) assimilation rate using a
spatially-explicit two-dimensional reaction diffusion model using a
porous medium approximation (Parkhurst 1994). Consider a two-dimensional
leaf where stomata occur on each surface in a regular sequence with
interstomatal distance \(U\). The main outcome we assessed is the
advantage of offsetting the position of stomata on each surface compared
to have stomata on the same \(x\) position on each surface. With these
assumptions, by symmetry, we only need to model two stomata, one abaxial
and one adaxial, from \(x = 0\) to \(x = U/2\) and from the adaxial
surface at \(y = 0\) to the abaxial surface at \(y = L\), the leaf
thickness. We arbitrarily set the adaxial stomate at \(x = 0\) and
toggled the abaxial stomata position between \(x = U/2\) (offset) or
\(x = 0\) (below adaxial stomate). The advantage of offsetting stomatal
position on each surface is the photosynthetic rate of the leaf with
offset stomata compared to that with stomata aligned in the same \(x\)
position:

\begin{equation} \label{eq:coordination_advantage}
  \text{coordination advantage} = \frac{A_\text{offset}}{A_\text{aligned}}
\end{equation}

We calculated the coordination advantage of a range of leaf thickness,
stomatal density, photosynthetic capacity, and light to understand when
offsetting stomatal position on each surface might deliver a significant
photosynthetic advantage.

\hypertarget{light-propogation-model}{%
\subsubsection{Light propogation model}\label{light-propogation-model}}

{[}NOTE: we should probably adjust light attenuation to be proportional
to the chlorophyll concentration which is one of the parameters in the
biochemical models.{]}

Irradiance at depth \(y\) in a leaf with thickness \(L\) is modeled
following Lloyd \emph{et al.} (1992):

\begin{equation}\label{eq:light_propogation}
  I(y) = 1.1 I_0 e ^ {-2.4 y / L}
\end{equation}

where \(I_0\) is photosynthetically active irradiance incident on the
adaxial leaf surface.

\hypertarget{biochemical-model}{%
\subsubsection{Biochemical model}\label{biochemical-model}}

All parameter symbols, units, descriptions, and values are described in
Table X below. Following Gutschick (1984), we modeled photosynthetic
rate per unit chlorophyll \(A_\text{chl}\) then calculated the
volumetric photosynthetic rate \(A_\text{volume}\) by multiplying by by
the chlorophyll concentration \(Q_\text{chl}\):

\[A_\text{volume} = A_\text{chl} Q_\text{chl}\] A description of the
model is given on page 553-556 of Gutschick (1984). The R code below is
how we implemented the model.

\begin{itemize}
\tightlist
\item
  \texttt{C\_m} is a vector CO\(_2\) concentrations in
  \(\text{mmol m}^{-3}\) at different positions within the leaf
\item
  \texttt{I\_m} is a vector of irradiances in
  \(\mu \text{mol m}^{-2} \text{s}^{-1}\) at different positions within
  the leaf (same order as \texttt{C\_m}).
\item
  \texttt{pars} is a list of parameters
\end{itemize}

\begin{Shaded}
\begin{Highlighting}[]
\NormalTok{pars }\OtherTok{=} \FunctionTok{list}\NormalTok{(}

    \AttributeTok{P =} \FunctionTok{set\_units}\NormalTok{(}\FloatTok{101.325}\NormalTok{, kPa), }\CommentTok{\# air pressure at sea level}
    \AttributeTok{temp =} \FunctionTok{set\_units}\NormalTok{(}\FloatTok{298.15}\NormalTok{, K), }\CommentTok{\# assume constant temperature}
    \AttributeTok{R\_gas =} \FunctionTok{set\_units}\NormalTok{(}\FloatTok{8.314}\NormalTok{, J}\SpecialCharTok{/}\NormalTok{K}\SpecialCharTok{/}\NormalTok{mol), }\CommentTok{\# ideal gas constant}

    \CommentTok{\# Calculations for C\_a and O ued below}
    \CommentTok{\# 21\% O2}
    \CommentTok{\# set\_units(0.21 * P / (R\_gas * temp), mol/m\^{}3)}
    \CommentTok{\# 415 ppm}
    \CommentTok{\# set\_units((415/1e6) * P / (R\_gas * temp), mmol/m\^{}3)}

    \CommentTok{\# Environmental}
    \AttributeTok{C\_a =} \FunctionTok{set\_units}\NormalTok{(}\FloatTok{16.96367}\NormalTok{, mmol}\SpecialCharTok{/}\NormalTok{m}\SpecialCharTok{\^{}}\DecValTok{3}\NormalTok{),}
    \AttributeTok{O =} \FunctionTok{set\_units}\NormalTok{(}\FloatTok{8.584027}\NormalTok{, mol}\SpecialCharTok{/}\NormalTok{m}\SpecialCharTok{\^{}}\DecValTok{3}\NormalTok{),}
    \AttributeTok{PAR =} \FunctionTok{set\_units}\NormalTok{(}\DecValTok{1000}\NormalTok{, umol}\SpecialCharTok{/}\NormalTok{m}\SpecialCharTok{\^{}}\DecValTok{2}\SpecialCharTok{/}\NormalTok{s),}

    \CommentTok{\# Biochemical}
    \AttributeTok{E\_t =} \FunctionTok{set\_units}\NormalTok{(}\FloatTok{0.01}\NormalTok{, mol}\SpecialCharTok{/}\NormalTok{mol),}
    \AttributeTok{eta\_t =} \FunctionTok{set\_units}\NormalTok{(}\FloatTok{0.59}\NormalTok{, }\DecValTok{1}\NormalTok{),}
    \AttributeTok{k\_c =} \FunctionTok{set\_units}\NormalTok{(}\DecValTok{20}\NormalTok{, mol}\SpecialCharTok{/}\NormalTok{mol}\SpecialCharTok{/}\NormalTok{s),}
    \AttributeTok{k\_o =} \FunctionTok{set\_units}\NormalTok{(}\FloatTok{4.2}\NormalTok{, mol}\SpecialCharTok{/}\NormalTok{mol}\SpecialCharTok{/}\NormalTok{s),}
    \AttributeTok{K\_c =} \FunctionTok{set\_units}\NormalTok{(}\FloatTok{0.0184}\NormalTok{, mol}\SpecialCharTok{/}\NormalTok{m}\SpecialCharTok{\^{}}\DecValTok{3}\NormalTok{),}
    \AttributeTok{K\_o =} \FunctionTok{set\_units}\NormalTok{(}\FloatTok{13.2}\NormalTok{, mol}\SpecialCharTok{/}\NormalTok{m}\SpecialCharTok{\^{}}\DecValTok{3}\NormalTok{),}
    \AttributeTok{Q\_chl =} \FunctionTok{set\_units}\NormalTok{(}\FloatTok{3.3}\NormalTok{, mol}\SpecialCharTok{/}\NormalTok{m}\SpecialCharTok{\^{}}\DecValTok{3}\NormalTok{),}
    \AttributeTok{R\_p =} \FunctionTok{set\_units}\NormalTok{(}\FloatTok{0.273}\NormalTok{, mol}\SpecialCharTok{/}\NormalTok{mol),}

    \CommentTok{\# Diffusivity of CO2 in leaf airspace. From Gutschick 1984. Should be updated and include temperature dependence, effect of cell packing, tortuosity, etc.}
    \AttributeTok{D\_mc =} \FunctionTok{set\_units}\NormalTok{(}\FloatTok{7e{-}7}\NormalTok{, m}\SpecialCharTok{\^{}}\DecValTok{2}\SpecialCharTok{/}\NormalTok{s)}

\NormalTok{)}

\CommentTok{\# strip units to speed up calculation}
\NormalTok{upars }\OtherTok{=}\NormalTok{ purrr}\SpecialCharTok{::}\FunctionTok{map}\NormalTok{(pars, }\SpecialCharTok{\textasciitilde{}}\NormalTok{ \{}\ControlFlowTok{if}\NormalTok{(}\FunctionTok{inherits}\NormalTok{(.x, }\StringTok{"units"}\NormalTok{)) \{}\FunctionTok{drop\_units}\NormalTok{(.x)\}\})}

\CommentTok{\# multiply/divide by 1e3 because C\_m is mmol and parameters are in mol}
\NormalTok{k\_vc }\OtherTok{=}\NormalTok{ upars[[}\StringTok{"k\_c"}\NormalTok{]] }\SpecialCharTok{/}\NormalTok{ (}\DecValTok{1} \SpecialCharTok{+}\NormalTok{ (}\FloatTok{1e3} \SpecialCharTok{*}\NormalTok{ upars[[}\StringTok{"K\_c"}\NormalTok{]] }\SpecialCharTok{/}\NormalTok{ C\_m) }\SpecialCharTok{*}
\NormalTok{                           (}\DecValTok{1} \SpecialCharTok{+}\NormalTok{ upars[[}\StringTok{"O"}\NormalTok{]] }\SpecialCharTok{/}\NormalTok{ upars[[}\StringTok{"K\_o"}\NormalTok{]])) }\CommentTok{\# [1/s]}
\CommentTok{\# Given light\_propagation assumptions, k = 2.4/L assuming no scattering}
\NormalTok{k\_i }\OtherTok{=} \FloatTok{2.4} \SpecialCharTok{/}\NormalTok{ upars[[}\StringTok{"leaf\_thickness"}\NormalTok{]] }\CommentTok{\# [1/um] \# I\textquotesingle{}m not sure sure this is right}
\NormalTok{b\_x }\OtherTok{=}\NormalTok{ k\_i }\SpecialCharTok{*}\NormalTok{ I\_m }\CommentTok{\# [mol/m\^{}3/s]}
\NormalTok{a\_x }\OtherTok{=}\NormalTok{ b\_x }\SpecialCharTok{/}\NormalTok{ upars[[}\StringTok{"Q\_chl"}\NormalTok{]] }\CommentTok{\# [1/s]}
\NormalTok{j }\OtherTok{=} \FloatTok{0.5} \SpecialCharTok{*}\NormalTok{ a\_x }\SpecialCharTok{*}\NormalTok{ upars[[}\StringTok{"eta\_t"}\NormalTok{]] }\CommentTok{\# [1/s] \# eqn 9}
\NormalTok{phi }\OtherTok{=}\NormalTok{ upars[[}\StringTok{"k\_o"}\NormalTok{]] }\SpecialCharTok{/}\NormalTok{ upars[[}\StringTok{"k\_c"}\NormalTok{]] }\SpecialCharTok{*}\NormalTok{ (upars[[}\StringTok{"O"}\NormalTok{]] }\SpecialCharTok{/}\NormalTok{ upars[[}\StringTok{"K\_o"}\NormalTok{]]) }\SpecialCharTok{/}
\NormalTok{  (}\FloatTok{1e{-}3} \SpecialCharTok{*}\NormalTok{ C\_m }\SpecialCharTok{/}\NormalTok{ upars[[}\StringTok{"K\_c"}\NormalTok{]]) }\CommentTok{\# [1]}
\NormalTok{v\_cj }\OtherTok{=}\NormalTok{ j }\SpecialCharTok{/}\NormalTok{ (}\DecValTok{4} \SpecialCharTok{+} \DecValTok{4} \SpecialCharTok{*}\NormalTok{ phi) }\CommentTok{\# [1/s]}
\NormalTok{v\_cp }\OtherTok{=}\NormalTok{ k\_vc }\SpecialCharTok{*}\NormalTok{ upars[[}\StringTok{"R\_p"}\NormalTok{]] }\CommentTok{\# [1/s]}
\NormalTok{v\_cr }\OtherTok{=} \FunctionTok{apply}\NormalTok{(}\FunctionTok{cbind}\NormalTok{(v\_cj, v\_cp), }\DecValTok{1}\NormalTok{, min) }\CommentTok{\# [1/s]}
\NormalTok{v\_c }\OtherTok{=} \FunctionTok{apply}\NormalTok{(}\FunctionTok{cbind}\NormalTok{(k\_vc }\SpecialCharTok{*}\NormalTok{ upars[[}\StringTok{"E\_t"}\NormalTok{]], v\_cr), }\DecValTok{1}\NormalTok{, min) }\CommentTok{\# [1/s]}
\NormalTok{v\_o }\OtherTok{=}\NormalTok{ phi }\SpecialCharTok{*}\NormalTok{ v\_c }\CommentTok{\# [1/s]}
\NormalTok{A\_chl }\OtherTok{=}\NormalTok{ v\_c }\SpecialCharTok{{-}} \FloatTok{0.5} \SpecialCharTok{*}\NormalTok{ v\_o }\CommentTok{\# [1/s] \# i.e mol CO2 / mol Chl / s}
\NormalTok{A\_volume }\OtherTok{=}\NormalTok{ A\_chl }\SpecialCharTok{*}\NormalTok{ upars[[}\StringTok{"Q\_chl"}\NormalTok{]] }\CommentTok{\# [mol/m\^{}3/s]}
\end{Highlighting}
\end{Shaded}

To model photosynthesis and CO\(_2\) transport within a two-dimensional
cross section of the leaf, we built a grid of nodes of dimensions leaf
thickness by half the interstomatal distance. Nodes represent the leaf
mesophyll where CO\(_2\) diffusion and assimilation occur.

\textbf{Table X} Glossary of mathematical symbols. The columns indicate
the mathematical Symbol used in the paper, the associated symbol used in
R scripts, scientific Units, and a verbal Description.

\begin{longtable}[]{@{}
  >{\raggedright\arraybackslash}p{(\columnwidth - 6\tabcolsep) * \real{0.12}}
  >{\raggedright\arraybackslash}p{(\columnwidth - 6\tabcolsep) * \real{0.10}}
  >{\raggedright\arraybackslash}p{(\columnwidth - 6\tabcolsep) * \real{0.34}}
  >{\raggedright\arraybackslash}p{(\columnwidth - 6\tabcolsep) * \real{0.44}}@{}}
\toprule
Symbol & Value(s) & Units & Description \\
\midrule
\endhead
Biochemical Parameters & & & \\
\(A_\text{volume}\) & NA & \(mol CO_2/m^3/s\) & volumetric assimilation
rate \\
\(A_\text{chl}\) & NA & \(mol CO2/mol Chl/s\) & assimilation rate per
mol chlorophyll \\
\(E_t\) & 0.01 & \(mol / mol Chl\) & rubisco octamer concentration per
chlorophyll \\
\(eta_t\) & 0.59 & unitless & quantum efficiency of photoexcitation
transfer to reaction centers \\
\(k_c\) & 20 & \(mol CO2/mol Rubisco/s\) at 25 C & maximal carboxylation
velocity of Rubisco \\
\(k_o\) & 4.2 & \(mol O2/mol Rubisco/s\) at 25 C & maximal oxygenation
velocity of Rubisco \\
\(K_c\) & 0.0184 & \(mol/m^3\) at 25 C & Michaelis constant for CO2 \\
\(K_o\) & 13.2 & \(mol/m^3\) at 25 C & Michaelis constant for O2 \\
\(J_\text{max}\) & 0.253 & \(mol e\text{-}/mol Chl/s\) & maximal
electron transport rate per mol Chl \\
\(Q_\text{chl}\) & 3.3 & \(mol/m^3\) & Chl volume concentration \\
\(v_c\) & NA & \(mol CO2/mol Chl/s\) & carboxylation rate \\
\(v_o\) & NA & \(mol O2/mol Chl/s\) & oxygenation rate \\
\(k_\text{vc}\) & NA & \(mol CO2/mol Rubisco/mol Chl\) & carboxylation
velocity per Rubisco octamer at ambient CO2 \\
\(R_p\) & 0.273 & \(mol RuBP/mol Chl\) & RuBP pool size \\
\(k_i\) & 2.4 & \(1/m^2\) or \(1/L\) & light attenuation coefficient \\
\(D_\text{mc}\) & 7e-7 & \(m^2/s\) & diffusivity of CO\(_2\) in
mesophyll airspace \\
Environmental Parameters & & & \\
\(P\) & 101.325 & \(kPa\) & air pressure at sea level \\
\(temp\) & 298.15 & \(K\) & temperature \\
\(R_\text{gas}\) & 8.314 & \(J/K/mol\) & ideal gas constant \\
\(C_a\) & 16.96367 & \(mmol CO_2/m^3\) & atmospheric CO\(_2\)
concentration \\
\(O\) & 8.584027 & \(mol O_2/m^3\) & atmospheric \(O_2\)
concentration \\
\(PAR\) & 1000 & \(umol photons/m^2/s\) & photosynthetically active
radiation \\
Variables & & & \\
\(U_s\) & .05, .10, .15\ldots{} & \(mm\) & interstomatal distance \\
\(T_l\) & .1, .2, .3\ldots{} & \(mm\) & leaf thickness \\
\(g_\text{sc}\) & 0.2, 0.3, 0.4\ldots{} & \(mol/m^2/s\) & stomatal
conductance \\
\bottomrule
\end{longtable}

\hypertarget{two-dimensional-model}{%
\subsection{Two-dimensional model}\label{two-dimensional-model}}

FYI - something is wrong with the model. I think the units are off
somewhere because the result is very sensitive to grid size and the
values often aren't reasonable. I'll need to troubleshoot more, but
maybe something will be obvious to you.

We used the \texttt{steady.2D()} function in the R package
\textbf{rootSolve}. The function needs a function to calculate the time
differential for node \(ij\), \(\frac{dC_{m,ij}}{dt}\). The R function
below uses the light propagation, biochemical, and diffusional model to
calculate a matrix of \(\frac{dC_{m,ij}}{dt}\) values:

\begin{Shaded}
\begin{Highlighting}[]
\NormalTok{diffusion2D }\OtherTok{\textless{}{-}} \ControlFlowTok{function}\NormalTok{ (time, state, pars, stomata\_offset) \{}

\NormalTok{  n\_row }\OtherTok{=}\NormalTok{ pars[[}\StringTok{"n\_row"}\NormalTok{]]}
\NormalTok{  n\_col }\OtherTok{=}\NormalTok{ pars[[}\StringTok{"n\_col"}\NormalTok{]]}

\NormalTok{  n\_node }\OtherTok{=}\NormalTok{ n\_row }\SpecialCharTok{*}\NormalTok{ n\_col}

\NormalTok{  node\_length\_m }\OtherTok{=}\NormalTok{ node\_length }\SpecialCharTok{*} \FloatTok{1e{-}6} \CommentTok{\# node\_length in [m]}

  \CommentTok{\# matrix of C\_m values}
\NormalTok{  C\_m\_mat }\OtherTok{=} \FunctionTok{matrix}\NormalTok{(}\AttributeTok{nrow =}\NormalTok{ n\_row, }\AttributeTok{ncol =}\NormalTok{ n\_col, state)}

  \CommentTok{\# Photosynthetic demand}
  \DocumentationTok{\#\# Light matrix}
\NormalTok{  I\_mat }\OtherTok{=} \FunctionTok{seq}\NormalTok{(}\DecValTok{0}\NormalTok{, }\AttributeTok{by =}\NormalTok{ pars[[}\StringTok{"node\_length"}\NormalTok{]], }\AttributeTok{length.out =}\NormalTok{ n\_row) }\SpecialCharTok{|}\ErrorTok{\textgreater{}}
    \FunctionTok{light\_propogation}\NormalTok{(pars[[}\StringTok{"leaf\_thickness"}\NormalTok{]], pars[[}\StringTok{"PAR"}\NormalTok{]]) }\SpecialCharTok{|}\ErrorTok{\textgreater{}}
    \FunctionTok{matrix}\NormalTok{(n\_row, }\AttributeTok{ncol =}\NormalTok{ n\_col)}

  \DocumentationTok{\#\# Biochemical model}
  \CommentTok{\# Units have to be correct for this to work, but unitless = TRUE should be faster}
\NormalTok{  A }\OtherTok{=} \FunctionTok{biochemical\_model}\NormalTok{(C\_m\_mat, I\_mat, }\AttributeTok{pars =}\NormalTok{ pars, }\AttributeTok{unitless =} \ConstantTok{TRUE}\NormalTok{)}

  \DocumentationTok{\#\# Photosynthesis matrix}
  \DocumentationTok{\#\# multiply by 1000 to convert from mol / m\^{}3 / s to mmol / m\^{}3 / s}
\NormalTok{  A\_mat }\OtherTok{=} \FunctionTok{matrix}\NormalTok{(}\AttributeTok{nrow =}\NormalTok{ n\_row, }\AttributeTok{ncol =}\NormalTok{ n\_col, }\FloatTok{1e3} \SpecialCharTok{*}\NormalTok{ A)}

  \CommentTok{\# CO2 diffusion}
\NormalTok{  C\_a }\OtherTok{=} \FunctionTok{drop\_units}\NormalTok{(pars[[}\StringTok{"C\_a"}\NormalTok{]])}
\NormalTok{  flux\_mat }\OtherTok{=} \FunctionTok{matrix}\NormalTok{(}\AttributeTok{nrow =}\NormalTok{ n\_row, }\AttributeTok{ncol =}\NormalTok{ n\_col, }\DecValTok{0}\NormalTok{)}
\NormalTok{  flux\_mat[}\DecValTok{1}\NormalTok{, }\DecValTok{1}\NormalTok{] }\OtherTok{=}\NormalTok{ pars[[}\StringTok{"g\_sc"}\NormalTok{]] }\SpecialCharTok{*}\NormalTok{ (C\_a }\SpecialCharTok{{-}}\NormalTok{ C\_m\_mat[}\DecValTok{1}\NormalTok{, }\DecValTok{1}\NormalTok{])}

  \DocumentationTok{\#\# 1. Flux through stomata}
  \ControlFlowTok{if}\NormalTok{ (stomata\_offset) \{}
\NormalTok{    flux\_mat[n\_row, n\_col] }\OtherTok{=}\NormalTok{ pars[[}\StringTok{"g\_sc"}\NormalTok{]] }\SpecialCharTok{*}\NormalTok{ (C\_a }\SpecialCharTok{{-}}\NormalTok{ C\_m\_mat[n\_row, n\_col])}
\NormalTok{  \} }\ControlFlowTok{else}\NormalTok{ \{}
\NormalTok{    flux\_mat[n\_row, }\DecValTok{1}\NormalTok{] }\OtherTok{=}\NormalTok{ pars[[}\StringTok{"g\_sc"}\NormalTok{]] }\SpecialCharTok{*}\NormalTok{ (C\_a }\SpecialCharTok{{-}}\NormalTok{ C\_m\_mat[n\_row, }\DecValTok{1}\NormalTok{])}
\NormalTok{  \}}

\NormalTok{  zero\_x }\OtherTok{\textless{}{-}} \FunctionTok{rep}\NormalTok{(}\DecValTok{0}\NormalTok{, n\_row)}
\NormalTok{  zero\_y }\OtherTok{\textless{}{-}} \FunctionTok{rep}\NormalTok{(}\DecValTok{0}\NormalTok{, n\_col)}

  \DocumentationTok{\#\# 2. Mesophyll flux; zero fluxes near boundaries}
\NormalTok{  flux\_above }\OtherTok{=} \FunctionTok{rbind}\NormalTok{(}\FunctionTok{rep}\NormalTok{(}\DecValTok{0}\NormalTok{, n\_col), C\_m\_mat[}\DecValTok{1}\SpecialCharTok{:}\NormalTok{(n\_row }\SpecialCharTok{{-}} \DecValTok{1}\NormalTok{), ] }\SpecialCharTok{{-}}\NormalTok{ C\_m\_mat[}\DecValTok{2}\SpecialCharTok{:}\NormalTok{n\_row, ])}
\NormalTok{  flux\_below }\OtherTok{=} \FunctionTok{rbind}\NormalTok{(C\_m\_mat[}\DecValTok{2}\SpecialCharTok{:}\NormalTok{n\_row, ] }\SpecialCharTok{{-}}\NormalTok{ C\_m\_mat[}\DecValTok{1}\SpecialCharTok{:}\NormalTok{(n\_row }\SpecialCharTok{{-}} \DecValTok{1}\NormalTok{), ], }\FunctionTok{rep}\NormalTok{(}\DecValTok{0}\NormalTok{, n\_col))}
\NormalTok{  flux\_left  }\OtherTok{=} \FunctionTok{cbind}\NormalTok{(}\FunctionTok{rep}\NormalTok{(}\DecValTok{0}\NormalTok{, n\_col), C\_m\_mat[, }\DecValTok{1}\SpecialCharTok{:}\NormalTok{(n\_col }\SpecialCharTok{{-}} \DecValTok{1}\NormalTok{)] }\SpecialCharTok{{-}}\NormalTok{ C\_m\_mat[, }\DecValTok{2}\SpecialCharTok{:}\NormalTok{n\_col])}
\NormalTok{  flux\_right }\OtherTok{=} \FunctionTok{cbind}\NormalTok{(C\_m\_mat[, }\DecValTok{2}\SpecialCharTok{:}\NormalTok{n\_col] }\SpecialCharTok{{-}}\NormalTok{ C\_m\_mat[, }\DecValTok{1}\SpecialCharTok{:}\NormalTok{(n\_col }\SpecialCharTok{{-}} \DecValTok{1}\NormalTok{)], }\FunctionTok{rep}\NormalTok{(}\DecValTok{0}\NormalTok{, n\_col))}

  \CommentTok{\# g\_mc [m/s] = D\_mc [m\^{}2/s] / node\_length [m]}
\NormalTok{  flux\_mat }\OtherTok{=}\NormalTok{ flux\_mat }\SpecialCharTok{+}\NormalTok{ pars[[}\StringTok{"g\_mc"}\NormalTok{]] }\SpecialCharTok{*}\NormalTok{ (flux\_above }\SpecialCharTok{+}\NormalTok{ flux\_below }\SpecialCharTok{+}\NormalTok{ flux\_left }\SpecialCharTok{+}\NormalTok{ flux\_right)}

    \FunctionTok{return}\NormalTok{(}\FunctionTok{list}\NormalTok{(}\FunctionTok{c}\NormalTok{(}\FunctionTok{as.vector}\NormalTok{(flux\_mat }\SpecialCharTok{/}\NormalTok{ node\_length\_m }\SpecialCharTok{{-}}\NormalTok{ A\_mat))))}

\NormalTok{\}}
\end{Highlighting}
\end{Shaded}

\hypertarget{global-results}{%
\subsection{Global Results}\label{global-results}}

Stomatal density of \emph{Arabidopsis thaliana} the 132 leaves measured
ranged from 12 to 93 (units) with high light leaves ranging from 93 to
55 (units), medium light from 15 to 35 (units), and low light from 12 to
42 (units). Leaves were amphistomatous with a mean stomatal ratio of
0.45.

\hypertarget{single-surface-results}{%
\subsection{Single Surface Results}\label{single-surface-results}}

If our hypotheses that natural selection will act to reduce the
diffusion path length of CO\(_2\) while minimizing stomatal number are
correct, then we would expect to find leaf surfaces with uniformly
distributed stomata. However, to the contrary, we find that though 57 of
the 132 (43.1\%) leaf surfaces were significantly more uniformly
dispersed than uniform random synthetic stomatal grids (\(\alpha\) =
0.05), none of the leaf surfaces exhibited perfectly uniform stomatal
patterning (dispersion index = 1) (Fig. \ref{fig:dispersion}).
Additionally, we hypothesized that as CO\(_2\) is more limiting to
photosynthesis under high light, stomata would be more uniformly
dispersed in plants grown in high light than plants grown in low and
medium light. The data also fail to support this hypothesis as there is
no strong, discernible trend between light and stomatal patterning.
Interestingly, adaxial leaf surfaces were more uniformly dispersed than
associated abaxial leaf surfaces across all light treatments
(F\(_\text{1,126}\) = 28.8; p-value \textless{} 0.001). Rather than
regulate stomatal patterning in response to light regimes, plants
respond by increasing stomatal density (Fig. \ref{fig:density}).
Stomatal density drastically increased in plants grown under high light,
validating the long held hypothesis that light strongly influences
stomatal density (F\(_\text{2,126}\) = 680.7; p-value \textless{}
0.001).

Across all light treatments and leaf surfaces, stomatal length and
stomatal area were weakly positively correlated, indicating some support
for our hypothesis that stomata are selected for an optimal operational
aperture (Fig. \ref{fig:length-area}).

\hypertarget{dual-surface-analysis}{%
\subsection{Dual Surface Analysis}\label{dual-surface-analysis}}

As evidence against our hypothesis that natural selection should favor
spatial coordination in the placement of stomata between abaxial and
adaxial leaf surfaces, we found no correlation between paired abaxial
<<<<<<< HEAD
and adaxial leaf surfaces. Light treatment had no effect on correlation
between surfaces. All but one abaxial, adaxial surface pairs were
independent (p-value \textless{} 0.05).
=======
and adaxial leaf surfaces (Fig. \ref{fig:dual-surface}). Light treatment
had no effect on correlation between surfaces (F\(_\text{2,63}\) = 2.28;
p-value = 0.11). All but one abaxial, adaxial surface pairs were
independent (\(\alpha\) = 0.05).
>>>>>>> a2d63857818f934454237c3ab5c736564b8f8ef9

\hypertarget{example-output}{%
\subsubsection{Example output}\label{example-output}}

We wrapped the above functions to adjust environmental and anatomical
variables. We're going to simulate over a grid of parameters, but here
is what the output looks like plotted:

\hypertarget{draft-text}{%
\subsubsection{DRAFT TEXT}\label{draft-text}}

During a model run, CO\(_2\) flux between neighboring nodes was
determined according to the mesophyll CO\(_2\) concentration (\(C_m\))
in each node, distance between cells, temperature, pressure, and leaf
porosity. Assimilation rate (\(A\)) in each node was determined by
\(A_\text{max}\), \(c_i\), irradiance, temperature, chlorophyll volume
concentration, Rubisco octamer concentration per mol chlorophyll, and
maximal oxygenation and carboxylation rates of Rubisco. Stomata were
placed either 1) directly on top of one another to represent
anticoordination or 2) offset from one another to represent coordination
between leaf surfaces. \(c_i\) and CO\(_2\) assimilation rate (A) were
set to 415ppm and 0 respectively and the model ran until \(c_i\) and A
reached equilibrium in each cell. Total A and average \(c_i\) were then
calculated for each parameter combination.

A was calculated in terms of \(A_\text{max}\) and \(c_i\) following Eq.
\ref{eq:eq4}

\begin{equation}\label{eq:eq4}
A = A_\text{max} * c_i / (c_i + K_c)
\end{equation}

where \(K_c\) is the Michaelis constant for CO\(_2\): 0.0184 mol
\(m^\text{-3}\).

\(A_\text{max}\) was calculated in terms of irradiance (i), light
saturated maximum A (\(A_\text{mm}\)), and quantum yield (\(\varphi\))
following Eq. \ref{eq:eq5}

\begin{equation}\label{eq:eq5}
A_\text{max} = A_\text{mm} * \varphi * i / (A_\text{mm} + \varphi * i)
\end{equation} \# Results

\hypertarget{single-surface-analysis-1}{%
\subsection{Single surface analysis}\label{single-surface-analysis-1}}

\begin{figure}[ht]
\includegraphics[width=\textwidth]{figures/single-surface.pdf}
\caption{Stomata are more dispersed than expected under the null model of uniform random position (dispersion index = 0) but far from a distribution that maximizes distance between stomata (dispersion index = 1). Significant differences between light treatments are indicated by asterisks according to analysis of variance followed by a post-hoc tukey honest significant difference test ($\alpha$ = 0.05).}
\label{fig:dispersion}
\end{figure}

\begin{figure}[ht]
\includegraphics[width = \textwidth]{figures/density.pdf}
\caption{Stomatal density is higher in plants grown under high light conditions. Significant differences between light treatments are indicated by asterisks according to analysis of variance followed by a post-hoc tukey honest significant difference test ($\alpha$ = 0.05).}
\label{fig:density}
\end{figure}

\begin{figure}[ht]
\includegraphics[width = \textwidth]{figures/tessellation-example.pdf}
\caption{Examples of synthetic and real leaf surfaces.  A) Uniform random synthetic leaf surface; B) Example of real leaf surface; C) Regularly distributed synthetic leaf surface. The zone defined by each stomate was calculated with voronoi tessellation and correlated with stomatal length in real leaves.}
\label{fig:example}
\end{figure}

\begin{figure}[ht]
\includegraphics[width = \textwidth]{figures/length-area.pdf}
\caption{Stomatal length and stomatal zone area. Lines of best fit computed using a bayesian generalized non-linear multilevel model. Each light level and leaf surface exhibits a unique, weakly positive relationship between stomatal zone area and length.}
\label{fig:length-area}
\end{figure}

\hypertarget{dual-surface-analysis-1}{%
\subsection{Dual surface analysis}\label{dual-surface-analysis-1}}

\begin{figure}[ht]
\includegraphics[width = \textwidth]{figures/ideal-amphi-grid.pdf}
\caption{Idealized amphistomatous stomatal grid with uniform stomatal patterning and perfect abaxial-adaxial coordination.}
\label{fig:amphi}
\end{figure}

\begin{figure}[ht]
\includegraphics[width = \textwidth]{figures/dual-surface.pdf}
\caption{Correlation between paired abaxial and adaxial leaf surfaces. Dashed line indicates no correlation. Weak positive correlations are not significantly different from null simulations. No differences in abaxial-adaxial correlation were observed between light levels according to an analysis of variance ($\alpha$ = 0.05).}
\label{fig:dual-surface}
\end{figure}

\hypertarget{discussion}{%
\section{Discussion}\label{discussion}}

You can cross-reference sections and subsections as follows: Section
\ref{materials-and-methods} and Section \ref{a-subsection}.

\textbf{\emph{Note:}} the last section in the document will be used as
the section title for the bibliography.

\hypertarget{references}{%
\section*{References}\label{references}}
\addcontentsline{toc}{section}{References}

\hypertarget{refs}{}
\begin{CSLReferences}{1}{0}
\leavevmode\hypertarget{ref-aalto_three-dimensional_2002}{}%
Aalto, T., and E. Juurola. 2002. {``A Three-Dimensional Model of {CO}
\(_{\textrm{2}}\) Transport in Airspaces and Mesophyll Cells of a Silver
Birch Leaf: {CO} \(_{\textrm{2}}\) Transport Inside a Birch Leaf.''}
\emph{Plant, Cell \& Environment} 25 (11): 1399--409.
\url{https://doi.org/10.1046/j.0016-8025.2002.00906.x}.

\leavevmode\hypertarget{ref-buckley_optimal_2017}{}%
Buckley, Thomas N, Lawren Sack, and Graham D Farquhar. 2017. {``Optimal
Plant Water Economy.''} \emph{Plant, Cell \& Environment} 40 (6):
881--96. \url{https://doi.org/10.1111/pce.12823}.

\leavevmode\hypertarget{ref-bussis_stomatal_2006}{}%
Büssis, Dirk, Uritza von Groll, Joachim Fisahn, and Thomas Altmann.
2006. {``Stomatal Aperture Can Compensate Altered Stomatal Density in
{Arabidopsis} Thaliana at Growth Light Conditions.''} \emph{Functional
Plant Biology} 33 (11): 1037. \url{https://doi.org/10.1071/FP06078}.

\leavevmode\hypertarget{ref-camargo_density_2011}{}%
Camargo, Miguel Angelo Branco, and Ricardo Antonio Marenco. 2011.
{``Density, Size and Distribution of Stomata in 35 Rainforest Tree
Species in {Central} {Amazonia}.''} \emph{Acta Amazonica} 41 (2):
205--12. \url{https://doi.org/10.1590/S0044-59672011000200004}.

\leavevmode\hypertarget{ref-carriqui_diffusional_2015}{}%
Carriquí, M., H. M. Cabrera, M. À. Conesa, R. E. Coopman, C. Douthe, J.
Gago, A. Gallé, et al. 2015. {``Diffusional Limitations Explain the
Lower Photosynthetic Capacity of Ferns as Compared with Angiosperms in a
Common Garden Study: {Photosynthetic} Comparison in Ferns and
Angiosperms.''} \emph{Plant, Cell \& Environment} 38 (3): 448--60.
\url{https://doi.org/10.1111/pce.12402}.

\leavevmode\hypertarget{ref-cowan_stomatal_1977}{}%
Cowan, IR, and GD Farquhar. 1977. {``{STOMATAL} {FUNCTION} {IN}
{RELATION} {TO} {LEAF} {METABOLISM} {AND} {ENVIRONMENT}.''}
\emph{STOMATAL FUNCTION IN RELATION TO LEAF METABOLISM AND ENVIRONMENT.}

\leavevmode\hypertarget{ref-croxdale_stomatal_2000}{}%
Croxdale, Judith L. 2000. {``Stomatal Patterning in Angiosperms.''}
\emph{American Journal of Botany} 87 (8): 1069--80.
\url{https://doi.org/10.2307/2656643}.

\leavevmode\hypertarget{ref-deans_optimization_2020}{}%
Deans, Ross M., Timothy J. Brodribb, Florian A. Busch, and Graham D.
Farquhar. 2020. {``Optimization Can Provide the Fundamental Link Between
Leaf Photosynthesis, Gas Exchange and Water Relations.''} \emph{Nature
Plants} 6 (9): 1116--25.
\url{https://doi.org/10.1038/s41477-020-00760-6}.

\leavevmode\hypertarget{ref-dow_physiological_2014}{}%
Dow, Graham J., Joseph A. Berry, and Dominique C. Bergmann. 2014. {``The
Physiological Importance of Developmental Mechanisms That Enforce Proper
Stomatal Spacing in \emph{{\textless{}}Span
Style="font-Variant:small-Caps;"{\textgreater{}}{A}{\textless{}}/Span{\textgreater{}}
Rabidopsis Thaliana}.''} \emph{New Phytologist} 201 (4): 1205--17.
\url{https://doi.org/10.1111/nph.12586}.

\leavevmode\hypertarget{ref-drake_two_2019}{}%
Drake, Paul L., Hugo J. Boer, Stanislaus J. Schymanski, and Erik J.
Veneklaas. 2019. {``Two Sides to Every Leaf: Water and {\textless{}}Span
Style="font-Variant:small-Caps;"{\textgreater{}}{CO}{\textless{}}/Span{\textgreater{}}
\(_{\textrm{2}}\) Transport in Hypostomatous and Amphistomatous
Leaves.''} \emph{New Phytologist} 222 (3): 1179--87.
\url{https://doi.org/10.1111/nph.15652}.

\leavevmode\hypertarget{ref-earles_beyond_2018}{}%
Earles, J. Mason, Guillaume Theroux-Rancourt, Adam B. Roddy, Matthew E.
Gilbert, Andrew J. McElrone, and Craig R. Brodersen. 2018. {``Beyond
{Porosity}: {3d} {Leaf} {Intercellular} {Airspace} {Traits} {That}
{Impact} {Mesophyll} {Conductance}.''} \emph{Plant Physiology} 178 (1):
148--62. \url{https://doi.org/10.1104/pp.18.00550}.

\leavevmode\hypertarget{ref-farquhar_biochemical_1980}{}%
Farquhar, G. D., S. von Caemmerer, and J. A. Berry. 1980. {``A
Biochemical Model of Photosynthetic {Co2} Assimilation in Leaves of {C3}
Species.''} \emph{Planta} 149 (1): 78--90.
\url{https://doi.org/10.1007/BF00386231}.

\leavevmode\hypertarget{ref-franks_physiological_2012}{}%
Franks, Peter J., Ilia J. Leitch, Elizabeth M. Ruszala, Alistair M.
Hetherington, and David J. Beerling. 2012. {``Physiological Framework
for Adaptation of Stomata to {CO} \(_{\textrm{2}}\) from Glacial to
Future Concentrations.''} \emph{Philosophical Transactions of the Royal
Society B: Biological Sciences} 367 (1588): 537--46.
\url{https://doi.org/10.1098/rstb.2011.0270}.

\leavevmode\hypertarget{ref-gay_influence_1975}{}%
Gay, A. P., and R. G. Hurd. 1975. {``{THE} {INFLUENCE} {OF} {LIGHT} {ON}
{STOMATAL} {DENSITY} {IN} {THE} {TOMATO}.''} \emph{New Phytologist} 75
(1): 37--46. \url{https://doi.org/10.1111/j.1469-8137.1975.tb01368.x}.

\leavevmode\hypertarget{ref-geisler_oriented_2000}{}%
Geisler, Matt, Jeanette Nadeau, and Fred D. Sack. 2000. {``Oriented
{Asymmetric} {Divisions} {That} {Generate} the {Stomatal} {Spacing}
{Pattern} in {Arabidopsis} {Are} {Disrupted} by the \emph{Too Many
Mouths} {Mutation}.''} \emph{The Plant Cell} 12 (11): 2075--86.
\url{https://doi.org/10.1105/tpc.12.11.2075}.

\leavevmode\hypertarget{ref-harrison_influence_2020}{}%
Harrison, Emily L., Lucia Arce Cubas, Julie E. Gray, and Christopher
Hepworth. 2020. {``The Influence of Stomatal Morphology and Distribution
on Photosynthetic Gas Exchange.''} \emph{The Plant Journal} 101 (4):
768--79. \url{https://doi.org/10.1111/tpj.14560}.

\leavevmode\hypertarget{ref-harwood_understanding_2021}{}%
Harwood, Richard, Guillaume Théroux‐Rancourt, and Margaret M Barbour.
2021. {``Understanding Airspace in Leaves: {\textless{}}Span
Style="font-Variant:small-Caps;"{\textgreater{}}{3D}{\textless{}}/Span{\textgreater{}}
Anatomy and Directional Tortuosity.''} \emph{Plant, Cell \&
Environment}, May, pce.14079. \url{https://doi.org/10.1111/pce.14079}.

\leavevmode\hypertarget{ref-jordan_using_2014}{}%
Jordan, Gregory J., Raymond J. Carpenter, and Timothy J. Brodribb. 2014.
{``Using Fossil Leaves as Evidence for Open Vegetation.''}
\emph{Palaeogeography, Palaeoclimatology, Palaeoecology} 395 (February):
168--75. \url{https://doi.org/10.1016/j.palaeo.2013.12.035}.

\leavevmode\hypertarget{ref-jordan_environmental_2015}{}%
Jordan, Gregory J., Raymond J. Carpenter, Anthony Koutoulis, Aina Price,
and Timothy J. Brodribb. 2015. {``Environmental Adaptation in Stomatal
Size Independent of the Effects of Genome Size.''} \emph{New
Phytologist} 205 (2): 608--17. \url{https://doi.org/10.1111/nph.13076}.

\leavevmode\hypertarget{ref-kaiser_metabolic_2016}{}%
Kaiser, Elias, Alejandro Morales, Jeremy Harbinson, Ep Heuvelink, Aina
E. Prinzenberg, and Leo F. M. Marcelis. 2016. {``Metabolic and
Diffusional Limitations of Photosynthesis in Fluctuating Irradiance in
{Arabidopsis} Thaliana.''} \emph{Scientific Reports} 6 (1): 31252.
\url{https://doi.org/10.1038/srep31252}.

\leavevmode\hypertarget{ref-lange_responses_1971}{}%
Lange, O. L., R. Lösch, E. -D. Schulze, and L. Kappen. 1971.
{``Responses of Stomata to Changes in Humidity.''} \emph{Planta} 100
(1): 76--86. \url{https://doi.org/10.1007/BF00386887}.

\leavevmode\hypertarget{ref-lee_diffusion_1964}{}%
Lee, Richard, and David M. Gates. 1964. {``{DIFFUSION} {RESISTANCE} {IN}
{LEAVES} {AS} {RELATED} {TO} {THEIR} {STOMATAL} {ANATOMY} {AND}
{MICRO}‐{STRUCTURE}.''} \emph{American Journal of Botany} 51 (9):
963--75. \url{https://doi.org/10.1002/j.1537-2197.1964.tb06725.x}.

\leavevmode\hypertarget{ref-lehmann_effects_2015}{}%
Lehmann, Peter, and Dani Or. 2015. {``Effects of Stomata Clustering on
Leaf Gas Exchange.''} \emph{New Phytologist} 207 (4): 1015--25.
\url{https://doi.org/10.1111/nph.13442}.

\leavevmode\hypertarget{ref-liu_scaling_2021}{}%
Liu, Congcong, Christopher D. Muir, Ying Li, Li Xu, Mingxu Li, Jiahui
Zhang, Hugo Jan de Boer, et al. 2021. {``Scaling Between Stomatal Size
and Density in Forest Plants.''} Preprint. Plant Biology.
\url{https://doi.org/10.1101/2021.04.25.441252}.

\leavevmode\hypertarget{ref-manter_ci_2004}{}%
Manter, D. K. 2004. {``A/{Ci} Curve Analysis Across a Range of Woody
Plant Species: Influence of Regression Analysis Parameters and Mesophyll
Conductance.''} \emph{Journal of Experimental Botany} 55 (408):
2581--88. \url{https://doi.org/10.1093/jxb/erh260}.

\leavevmode\hypertarget{ref-mcadam_linking_2016}{}%
McAdam, Scott A. M., and Timothy J. Brodribb. 2016. {``Linking {Turgor}
with {ABA} {Biosynthesis}: {Implications} for {Stomatal} {Responses} to
{Vapor} {Pressure} {Deficit} Across {Land} {Plants}.''} \emph{Plant
Physiology} 171 (3): 2008--16.
\url{https://doi.org/10.1104/pp.16.00380}.

\leavevmode\hypertarget{ref-morison_lateral_2005}{}%
Morison, James I. L., Emily Gallouët, Tracy Lawson, Gabriel Cornic,
Raphaèle Herbin, and Neil R. Baker. 2005. {``Lateral {Diffusion} of {CO}
{\textless{}}Sub{\textgreater{}}2{\textless{}}sub{\textgreater{}} in
{Leaves} {Is} {Not} {Sufficient} to {Support} {Photosynthesis}.''}
\emph{Plant Physiology} 139 (1): 254--66.
\url{https://doi.org/10.1104/pp.105.062950}.

\leavevmode\hypertarget{ref-mott_adaptive_1982}{}%
Mott, Keith A., Arthur C. Gibson, and James W. O'Leary. 1982. {``The
Adaptive Significance of Amphistomatic Leaves.''} \emph{Plant, Cell \&
Environment} 5 (6): 455--60.
\url{https://doi.org/10.1111/1365-3040.ep11611750}.

\leavevmode\hypertarget{ref-mott_amphistomy_1991}{}%
Mott, Keith A., and Odette Michaelson. 1991. {``{AMPHISTOMY} {AS} {AN}
{ADAPTATION} {TO} {HIGH} {LIGHT} {INTENSITY} {IN} {AMBROSIA}
{CORDIFOLIA} ({COMPOSITAE}).''} \emph{American Journal of Botany} 78
(1): 76--79. \url{https://doi.org/10.1002/j.1537-2197.1991.tb12573.x}.

\leavevmode\hypertarget{ref-muir_light_2018}{}%
Muir, Christopher D. 2018. {``Light and Growth Form Interact to Shape
Stomatal Ratio Among {British} Angiosperms.''} \emph{New Phytologist}
218 (1): 242--52. \url{https://doi.org/10.1111/nph.14956}.

\leavevmode\hypertarget{ref-muir_is_2019}{}%
Muir, Christopher D. 2019. {``Is {Amphistomy} an {Adaptation} to {High}
{Light}? {Optimality} {Models} of {Stomatal} {Traits} Along {Light}
{Gradients}.''} \emph{Integrative and Comparative Biology} 59 (3):
571--84. \url{https://doi.org/10.1093/icb/icz085}.

\leavevmode\hypertarget{ref-muir_stomatal_2020}{}%
Muir, Christopher D. 2020. {``A {Stomatal} {Model} of {Anatomical}
{Tradeoffs} {Between} {Gas} {Exchange} and {Pathogen} {Colonization}.''}
\emph{Frontiers in Plant Science} 11 (October): 518991.
\url{https://doi.org/10.3389/fpls.2020.518991}.

\leavevmode\hypertarget{ref-papanatsiou_stomatal_2017}{}%
Papanatsiou, Maria, Anna Amtmann, and Michael R. Blatt. 2017.
{``Stomatal Clustering in {Begonia} Associates with the Kinetics of Leaf
Gaseous Exchange and Influences Water Use Efficiency.''} \emph{Journal
of Experimental Botany} 68 (9): 2309--15.
\url{https://doi.org/10.1093/jxb/erx072}.

\leavevmode\hypertarget{ref-parkhurst_adaptive_1978}{}%
Parkhurst, David F. 1978. {``The {Adaptive} {Significance} of {Stomatal}
{Occurrence} on {One} or {Both} {Surfaces} of {Leaves}.''} \emph{Journal
of Ecology} 66 (2): 367--83. \url{https://doi.org/10.2307/2259142}.

\leavevmode\hypertarget{ref-parkhurst_intercellular_1990}{}%
Parkhurst, David F., and Keith A. Mott. 1990. {``Intercellular
{Diffusion} {Limits} to {CO} \(_{\textrm{2}}\) {Uptake} in {Leaves}:
{Studies} in {Air} and {Helox}.''} \emph{Plant Physiology} 94 (3):
1024--32. \url{https://doi.org/10.1104/pp.94.3.1024}.

\leavevmode\hypertarget{ref-pieruschka_lateral_2005}{}%
Pieruschka, R. 2005. {``Lateral Gas Diffusion Inside Leaves.''}
\emph{Journal of Experimental Botany} 56 (413): 857--64.
\url{https://doi.org/10.1093/jxb/eri072}.

\leavevmode\hypertarget{ref-pieruschka_lateral_2006}{}%
Pieruschka, Roland, Ulrich Schurr, Manfred Jensen, Wilfried F. Wolff,
and Siegfried Jahnke. 2006. {``Lateral Diffusion of {CO}
\(_{\textrm{2}}\) from Shaded to Illuminated Leaf Parts Affects
Photosynthesis Inside Homobaric Leaves.''} \emph{New Phytologist} 169
(4): 779--88. \url{https://doi.org/10.1111/j.1469-8137.2005.01605.x}.

\leavevmode\hypertarget{ref-roddy_scaling_2020}{}%
Roddy, Adam B., Guillaume Théroux-Rancourt, Tito Abbo, Joseph W.
Benedetti, Craig R. Brodersen, Mariana Castro, Silvia Castro, et al.
2020. {``The {Scaling} of {Genome} {Size} and {Cell} {Size} {Limits}
{Maximum} {Rates} of {Photosynthesis} with {Implications} for
{Ecological} {Strategies}.''} \emph{International Journal of Plant
Sciences} 181 (1): 75--87. \url{https://doi.org/10.1086/706186}.

\leavevmode\hypertarget{ref-royer_stomatal_2001}{}%
Royer, D. L. 2001. {``Stomatal Density and Stomatal Index as Indicators
of Paleoatmospheric {Co2} Concentration.''} \emph{Review of Palaeobotany
and Palynology} 114 (1-2): 1--28.
\url{https://doi.org/10.1016/S0034-6667(00)00074-9}.

\leavevmode\hypertarget{ref-sachs_developmental_1974}{}%
Sachs, T. 1974. {``The {Developmental} {Origin} of {Stomata} {Pattern}
in {Crinum}.''} \emph{Botanical Gazette} 135 (4): 314--18.
\url{https://doi.org/10.1086/336767}.

\leavevmode\hypertarget{ref-sack_developmental_2016}{}%
Sack, Lawren, and Thomas N. Buckley. 2016. {``The {Developmental}
{Basis} of {Stomatal} {Density} and {Flux}.''} \emph{Plant Physiology}
171 (4): 2358--63. \url{https://doi.org/10.1104/pp.16.00476}.

\leavevmode\hypertarget{ref-schoch_dependence_1980}{}%
Schoch, Paul-G., Claude Zinsou, and Monique Sibi. 1980. {``Dependence of
the {Stomatal} {Index} on {Environmental} {Factors} During {Stomatal}
{Differentiation} in {Leaves} of \emph{{Vigna} Sinensis} {L}.: 1.
{EFFECT} {OF} {LIGHT} {INTENSITY}.''} \emph{Journal of Experimental
Botany} 31 (5): 1211--16. \url{https://doi.org/10.1093/jxb/31.5.1211}.

\leavevmode\hypertarget{ref-sperry_predicting_2017}{}%
Sperry, John S., Martin D. Venturas, William R. L. Anderegg, Maurizio
Mencuccini, D. Scott Mackay, Yujie Wang, and David M. Love. 2017.
{``Predicting Stomatal Responses to the Environment from the
Optimization of Photosynthetic Gain and Hydraulic Cost: {A} Stomatal
Optimization Model.''} \emph{Plant, Cell \& Environment} 40 (6):
816--30. \url{https://doi.org/10.1111/pce.12852}.

\leavevmode\hypertarget{ref-woodward_stomatal_1987}{}%
Woodward, F. I. 1987. {``Stomatal Numbers Are Sensitive to Increases in
{Co2} from Pre-Industrial Levels.''} \emph{Nature} 327 (6123): 617--18.
\url{https://doi.org/10.1038/327617a0}.

\leavevmode\hypertarget{ref-yi_gan_stomatal_2010}{}%
Yi Gan, Lei Zhou, Zhong-Ji Shen, Zhu-Xia Shen, Yi-Qiong Zhang, and
Gen-Xuan Wang. 2010. {``Stomatal Clustering, a New Marker for
Environmental Perception and Adaptation in Terrestrial Plants.''}
\emph{Botanical Studies} 51 (3): 325--36.
\url{https://search.ebscohost.com/login.aspx?direct=true\&db=a9h\&AN=60102322\&site=ehost-live}.

\end{CSLReferences}


\begin{notes}[Acknowledgements]
This is an acknowledgement.

It consists of two paragraphs.
\end{notes}




\end{document}
