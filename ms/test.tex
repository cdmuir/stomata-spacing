% Options for packages loaded elsewhere
\PassOptionsToPackage{unicode}{hyperref}

% Set document class options
\documentclass[webpdf,large,modern,unnumsec,namedate]{oup-authoring-template}

% one column
\onecolumn

%\usepackage{showframe}

% line numbers
\usepackage{lineno}
\linenumbers

% use upquote if available, for straight quotes in verbatim environments
\IfFileExists{upquote.sty}{\usepackage{upquote}}{}

% From Pandoc template for its feature
\usepackage{xcolor}
\usepackage{hyperref}

\hypersetup{
  pdftitle={Does stomatal patterning in amphistomatous leaves minimize the CO\_2 diffusion path length within leaves?},
  pdfkeywords={amphistomy, Arabidopsis thaliana, CO\(_2\)
diffusion, finite element method, optimality, photosynthesis, stomata},
  breaklinks=true,
  bookmarks=true,
  hidelinks,
  pdfcreator={LaTeX via pandoc}}



% tightlist command for lists without linebreak
\providecommand{\tightlist}{%
  \setlength{\itemsep}{0pt}\setlength{\parskip}{0pt}}




% Counters for addresses and footnotes
\newcounter{correspcnt} % For author footnotes
\renewcommand*{\thecorrespcnt}{\fnsymbol{correspcnt}}
\newcounter{addrcnt} % For author addresses

% Macros for dealing with affiliations, footnotes, etc.
\makeatletter

\def\MyNewLabel#1#2#3{\expandafter\gdef\csname #1@#2\endcsname{#3}}

\def\MyRef#1#2{\@ifundefined{#1@#2}{???}{\csname #1@#2\endcsname}}

\newcommand*\ifcounter[1]{%
  \ifcsname c@#1\endcsname
    \expandafter\@firstoftwo
  \else
    \expandafter\@secondoftwo
  \fi
}

\newcommand*\addrlblbycode[1]{%
  \ifcounter{ADDRLBL@#1}
    {}
    {\refstepcounter{addrcnt}\newcounter{ADDRLBL@#1}\setcounter{ADDRLBL@#1}{\value{addrcnt}}}%
    \arabic{ADDRLBL@#1}%
}

\newcommand*\addrbycode[1]{%
  \ifcounter{ADDR@#1}
    {}
    {\newcounter{ADDR@#1}%
     \address[\addrlblbycode{#1}]{\MyRef{ADDRTXT}{#1}}}%
}

\newcommand*\corresplblbycode[1]{%
  \ifcounter{CORRESPLBL@#1}
    {}
    {\refstepcounter{correspcnt}\newcounter{CORRESPLBL@#1}\setcounter{CORRESPLBL@#1}{\value{correspcnt}}}%
    \fnsymbol{CORRESPLBL@#1}%
}

\newcommand*\correspbycode[1]{%
  \ifcounter{CORRESP@#1}
    {}
    {\newcounter{CORRESP@#1}%
     \corresp[\corresplblbycode{#1}]{\MyRef{CORRESPTXT}{#1}}}%
}

\makeatother

% Add missing \city command mentioned in documentation but absent from cls
\providecommand\city[1]{#1}

% Create labels for Addresses if the are given in Elsevier format
   \MyNewLabel{ADDRTXT}{UHM}{%
  School of Life Sciences, University of Hawaiʻi at Mānoa, Honolulu, HI
96822%
 }
   \MyNewLabel{ADDRTXT}{CU}{%
  Ecology and Evolutionary Biology, University of Colorado, Boulder, CO
80309%
 }
   \MyNewLabel{ADDRTXT}{NIAB}{%
  Department of Crop Science and Production Systems, NIAB, Cambridge,
CB3 0LE, UK%
 }
   \MyNewLabel{ADDRTXT}{UCD}{%
  Department of Plant Sciences, University of California, Davis, CA
95616%
 }
   \MyNewLabel{ADDRTXT}{UWM}{%
  Department of Botany, University of Wisconsin, Madison, WI 53706%
 }

% Create labels for Footnotes if they are given in Elsevier format

% Pandoc header-include feature
\usepackage{setspace}
\onehalfspacing
\usepackage[nomarkers,tablesfirst]{endfloat}
\usepackage{hyperref}
\renewcommand{\figureautorefname}{Fig.}
\usepackage[detect-none]{siunitx}
\sisetup{range-phrase = \text{--}}
\usepackage{caption}
\usepackage{newunicodechar,graphicx}
\DeclareRobustCommand{\okina}{\raisebox{\dimexpr\fontcharht\font`A-\height}{\scalebox{0.8}{`}}}
\newunicodechar{ʻ}{\okina}
% Pandoc header-include feature
\usepackage{booktabs}

\begin{document}

\journaltitle{Journal Title Here}
\DOI{DOI HERE}
\copyrightyear{YYYY}
\pubyear{YYYY}
\access{Advance Access Publication Date: Day Month Year}
\appnotes{Paper}

\firstpage{1}



\title[]{Does stomatal patterning in amphistomatous leaves minimize the
CO\(_2\) diffusion path length within leaves?}

\newcounter{thisauthcorresp} % For storage if author is corresponding author
\newcounter{thisauththanks} % For storage if author has thanks



\author[%
\addrlblbycode{UHM},\addrlblbycode{CU}%
,\refstepcounter{correspcnt}\setcounter{thisauthcorresp}{\value{correspcnt}}\fnsymbol{thisauthcorresp}%
%
%
]{Jacob L. Watts}

\addrbycode{UHM}
\addrbycode{CU}

\corresp[\fnsymbol{thisauthcorresp}]{Corresponding author. \href{mailto:Jacob.Watts-1@colorado.edu}{\nolinkurl{Jacob.Watts-1@colorado.edu}}}




\author[%
\addrlblbycode{NIAB}%
%
%
%
]{Graham J. Dow}

\addrbycode{NIAB}






\author[%
\addrlblbycode{UCD}%
%
%
%
]{Thomas N. Buckley}

\addrbycode{UCD}






\author[%
\addrlblbycode{UHM},\addrlblbycode{UWM}%
%
%
%
]{Christopher D. Muir}

\addrbycode{UHM}
\addrbycode{UWM}






% Add author mark
\authormark{Jacob L. Watts et al.}

\received{Date}{0}{Year}
\revised{Date}{0}{Year}
\accepted{Date}{0}{Year}

%\editor{Associate Editor: Name}

\abstract{
Photosynthesis is co-limited by multiple factors depending on the plant
and its environment. These include biochemical rate limitations,
internal and external water potentials, temperature, irradiance, and
carbon dioxide (CO\(_2\)). Amphistomatous leaves have stomata on both
abaxial and adaxial leaf surfaces. This feature is considered an
adaptation to alleviate CO\(_2\) diffusion limitations in productive
environments where other factors are not limiting as the diffusion path
length from stomate to chloroplast is effectively halved. Plants can
also reduce CO\(_2\) limitations through other aspects of optimal
stomatal anatomy: stomatal density, distribution, patterning, and size.
A number of studies have demonstrated that stomata are overdispersed on
a single leaf surface; however, much less is known about stomatal
anatomy in amphistomatous leaves, especially the coordination between
leaf surfaces, despite their prevelance in nature and near ubiquity
among crop species. Here we use novel spatial statistics based on
simulations and photosynthesis modeling to test hypotheses about how
amphistomatous plants may optimize CO\(_2\) limitations in the model
angiosperm \emph{Arabidopsis thaliana} grown in different light
environments. We find that 1) stomata are overdispersed, but not ideally
dispersed, on both leaf surfaces across all light treatments; 2) abaxial
and adaxial leaf surface patterning are independent; and 3) the
theoretical improvements to photosynthesis from abaxial-adaxial stomatal
coordination are miniscule (\(\ll 1\)\%) across the range of feasible
parameter space. However, we also find that 4) stomatal size is
correlated with the mesophyll volume that it supplies with CO\(_2\),
suggesting that plants may optimize CO\(_2\) diffusion limitations
through alternative pathways other than ideal, uniform stomatal spacing.
We discuss the developmental, physical, and evolutionary constraits
which may prohibit plants from reaching the theoretical adaptive peak of
uniform stomatal spacing and inter-surface stomatal coordination. These
findings contribute to our understanding of variation in the anatomy of
amphistomatous leaves.}

\keywords{amphistomy; Arabidopsis thaliana; CO\(_2\) diffusion; finite
element method; optimality; photosynthesis; stomata}


\maketitle


\hypertarget{introduction}{%
\section{Introduction}\label{introduction}}

Stomatal anatomy (e.g.~size, density, distribution, and patterning) and
movement regulate gas exchange during photosynthesis, namely CO\(_2\)
assimilation and water loss through transpiration. Since waxy cuticles
are mostly impermeable to CO\(_2\) and H\(_2\)O, stomata are the primary
entry and exit points through which gas exchange occurs despite making
up a small percentage of the leaf area \citep{lange_responses_1971}.
Stomata consist of two guard cells which open and close upon changes in
turgor pressure or hormonal cues \citep{mcadam_linking_2016}. The
stomatal pore leads to an internal space known as the substomatal cavity
where gases contact the mesophyll. Once in the mesophyll, CO\(_2\)
diffuses throughout a network of intercellular air space (IAS) and into
mesophyll cells where CO\(_2\) assimilation (\(A\)) occurs within the
chloroplasts \citep{lee_diffusion_1964}. Stomatal conductance and
transpiration are determined by numerous environmental and anatomical
parameters such as vapor pressure deficit (VPD), irradiance,
temperature, wind speed, leaf water potential, IAS geometry, mesophyll
cell anatomy, and stomatal anatomy.

\section{Competing interests}

The authors declare no competing interests.

\section{Author contributions statement}

JLW and CDM conceived of the project, analyzed data, and wrote the
manuscript. GJD provided data. TNB contributed to model development and
helped edit the manuscript.



\bibliographystyle{abbrvnat}
\bibliography{stomata-spacing.bib}

%% Author bio-pics with images


\end{document}
